\documentclass[a4,11pt]{aleph-notas}
% Se puede ver la documentación aquí: 
% https://github.com/alephsub0/LaTeX_aleph-notas

% -- Paquetes adicionales 
\usepackage{enumitem}
\usepackage{array,booktabs,multirow,makecell}
\usepackage{colortbl}
\usepackage{longtable}
\usepackage{url}
\usepackage{pdflscape}

% -- Datos 
\institucion{Escuela de Ciencias Físicas y Matemática}
\carrera{Medicina Veterinaria}
\asignatura{Matemática I*}
\tema{Cronograma y actividades}
\autor{Andrés Merino}
\fecha{Semestre 2024-2}

\logouno[0.14\textwidth]{Logos/logoPUCE_04_ac}
\definecolor{colortext}{HTML}{0030A1}
\definecolor{colordef}{HTML}{0030A1}
\fuente{montserrat}


% -- Comandos para tablas
\newcolumntype{C}[1]{>{\hspace{0pt}\centering\arraybackslash}p{#1}}
\newcolumntype{L}[1]{>{\raggedright\arraybackslash}p{#1}}

\definecolor{verde}{RGB}{0, 255, 127}
\definecolor{celeste}{RGB}{68,195,218}

\begin{document}
\addtolength{\headheight}{1.8\baselineskip}
\addtolength{\voffset}{-1.5\baselineskip}

\encabezado

%%%%%%%%%%%%%%%%%%%%%%%%%%%%%%%%%%%%%%%%
\section{Resultados de aprendizaje} 
%%%%%%%%%%%%%%%%%%%%%%%%%%%%%%%%%%%%%%%%

\begin{itemize}[leftmargin=*]
\item 
    \textbf{RdA 1:} Comprender las nociones de lógica matemática, teoría de conjuntos, números reales y los conceptos fundamentales de estadística en diversos contextos.
    \begin{itemize}[leftmargin=*]
        \item \textbf{Criterio 1.1:} Comprende los conceptos fundamentales de lógica matemática, teoría de conjuntos y números reales aplicables en su campo.
        \item \textbf{Criterio 1.2:} Reconoce los conceptos fundamentales de estadística, incluyendo población, muestra, parámetros y variables.
    \end{itemize}
\item 
    \textbf{RdA 2:} Resolver ecuaciones e inecuaciones, lineales y cuadráticas, de una y varias variables, mediante la utilización de transformaciones algebraicas y las propiedades de los números reales en problemas matemáticos de diversos contextos.
    \begin{itemize}[leftmargin=*]
        \item \textbf{Criterio 2.1:} Resuelve ecuaciones e inecuaciones lineales y cuadráticas de una variable utilizando las propiedades de los números reales.
        \item \textbf{Criterio 2.2:} Resuelve sistemas de ecuaciones e inecuaciones de varias variables, aplicando métodos algebraicos y geométricos adecuados.
    \end{itemize}
\item
    \textbf{RdA 3:} Aplicar los conceptos de matemática básica y estadística básica en la resolución de problemas prácticos en diversos contextos.
    \begin{itemize}[leftmargin=*]
        \item \textbf{Criterio 3.1:} Aplica los principios de lógica matemática, teoría de conjuntos y números reales para la resolución de problemas prácticos.
        \item \textbf{Criterio 3.2:} Analiza soluciones obtenidas de ecuaciones e inecuaciones en problemas de diversos contextos, mediante la evaluación de la validez y pertinencia de los resultados.
        \item \textbf{Criterio 3.3:} Utiliza conceptos estadísticos básicos para el análisis de datos relevantes en la práctica, como tasas de crecimiento, dosificación de medicamentos, análisis de poblaciones, etc.
    \end{itemize}
\end{itemize}

%%%%%%%%%%%%%%%%%%%%%%%%%%%%%%%%%%%%%%%%
\section{Contenidos generales} 
%%%%%%%%%%%%%%%%%%%%%%%%%%%%%%%%%%%%%%%%

\begin{itemize}
\item 
    Lógica y teoría de conjuntos
\item 
    Introducción a Funciones
\item 
    Expresiones Algebraicas
\item 
    Ecuaciones e inecuaciones de una variable
\item 
    Sistemas de ecuaciones e inecuaciones de dos variables
\item 
    Conceptos de estadística y probabilidad
\end{itemize}

%%%%%%%%%%%%%%%%%%%%%%%%%%%%%%%%%%%%%%%%
\section{Actividades de evaluación} 
%%%%%%%%%%%%%%%%%%%%%%%%%%%%%%%%%%%%%%%%

\begin{itemize}[leftmargin=*]
    \item \textbf{Criterio 1.1}
        \begin{itemize}[leftmargin=*]
            \item \textbf{Cuestionario en Línea 1 (50\%):} Evaluará la comprensión de los conceptos fundamentales de lógica matemática y teoría de conjuntos, a través de preguntas de opción múltiple, calculadas y ce completación.
            \item \textbf{Taller 1 (50\%):} Consolidará los conceptos mediante la resolución colaborativa de problemas prácticos. El objetivo es aplicar las operaciones lógicas y el manejo de conjuntos en situaciones guiadas.
        \end{itemize}

    \item \textbf{Criterio 1.2}
        \begin{itemize}[leftmargin=*]
            \item \textbf{Cuestionario en Línea 2 (100\%):} Evaluará el reconocimiento de conceptos estadísticos, incluyendo población, muestra, parámetros y variables, a través de preguntas de opción múltiple, calculadas y ce completación.
        \end{itemize}

    \item \textbf{Criterio 2.1}
        \begin{itemize}[leftmargin=*]
            \item \textbf{Examen Escrito 1 (100\%):} Evaluará la resolución de ecuaciones e inecuaciones lineales y cuadráticas de una variable, con ejercicios de interpretación y aplicación práctica.
        \end{itemize}

    \item \textbf{Criterio 2.2}
        \begin{itemize}[leftmargin=*]
            \item \textbf{Examen Escrito 2 (100\%):} Evaluará la resolución de sistemas de ecuaciones e inecuaciones de varias variables, utilizando métodos algebraicos y gráficos para encontrar soluciones.
        \end{itemize}

    \item \textbf{Criterio 3.1}
        \begin{itemize}[leftmargin=*]
            \item \textbf{Reto 1 (100\%):} Aplicará los principios de lógica matemática, teoría de conjuntos y números reales en la resolución de un problema práctico. Se trabajará en grupos, con énfasis en la colaboración y la aplicación.
        \end{itemize}

    \item \textbf{Criterio 3.2}
        \begin{itemize}[leftmargin=*]
            \item \textbf{Examen Escrito 3 (100\%):} Evaluará la capacidad de plantar ecuaciones e inecuaciones, resolverlas y analizar las soluciones obtenidas, verificando su validez y pertinencia en problemas prácticos.
        \end{itemize}

    \item \textbf{Criterio 3.3}
        \begin{itemize}[leftmargin=*]
            \item \textbf{Reto 2 (100\%):} Evaluará el uso de conceptos estadísticos para el análisis de datos aplicados, como tasas de crecimiento, dosificación de medicamentos o análisis de poblaciones, mediante trabajo en grupo.
        \end{itemize}
\end{itemize}
\begin{landscape}


%%%%%%%%%%%%%%%%%%%%%%%%%%%%%%%%%%%%%%%%
\section{Cronograma de Desarrollo del Curso} %%%%%%%%%%%%%%%%%%%%%%%%%%%%%%%%%%%%%%%%

\begin{center}\small
\setlength{\extrarowheight}{0ex}
\setlength{\belowrulesep}{.6ex}
\begin{longtable}{cccL{13cm}L{6cm}}
    \toprule
    &&\thead{Fecha}&\thead{Detalle de contenido} & \thead{Observación} \\
    \midrule
  \endfirsthead
    \multicolumn{4}{l}{\footnotesize \ldots viene de la página anterior}\\
    \toprule
    &&\thead{Fecha}&\thead{Detalle de contenido} & \thead{Observación} \\
    \midrule
  \endhead
        \bottomrule  \multicolumn{4}{r}{\footnotesize Continúa en la siguiente página\ldots}
  \endfoot
        \bottomrule
  \endlastfoot
1	&	1	&	01-oct	&	Introducción al Curso, Conceptos Básicos de Lógica	&		\\	
	&	2	&	03-oct	&	Proposiciones y Conectivos Lógicos, Tablas de Verdad	&		\\ \midrule	
2	&	3	&	08-oct	&	Operaciones Lógicas y su Aplicación en Problemas	&		\\	
	&	4	&	10-oct	&	Teoría de Conjuntos: Conceptos y Notaciones	&	Envío del Reto 1 (Criterio 3.1)	\\ \midrule	
3	&	5	&	15-oct	&	Operaciones con Conjuntos	&		\\	\rowcolor{celeste!50}
	&	6	&	17-oct	&	Cuestionario en Línea 1 (Criterio 1.1)	&	Evaluación	\\ \midrule	
4	&	7	&	22-oct	&	Funciones: Definición y Propiedades	&		\\	
	&	8	&	24-oct	&	Aplicaciones de Funciones	&	Aula invertida	\\ \midrule	
5	&	9	&	29-oct	&	Simplificación de Expresiones Algebraicas	&	Entrega del Reto 1 (Criterio 3.1)	\\	
	&	10	&	31-oct	&	Factorización y Expansión de Expresiones Algebraicas	&		\\ \midrule	\rowcolor{celeste!50}
6	&	11	&	05-nov	&	Taller 1 (Criterio 1.1)	&	Evaluación	\\	
	&	12	&	07-nov	&	Ecuaciones Lineales: Definición y Resolución	&		\\ \midrule	
7	&	13	&	12-nov	&	Ecuaciones Cuadráticas: Solución Analítica	&	Aula invertida	\\	
	&	14	&	14-nov	&	Inecuaciones de una Variable: Solución y Representación Gráfica	&		\\ \midrule	
8	&	15	&	19-nov	&	Ejercicios de Ecuaciones e Inecuaciones	&		\\	\rowcolor{celeste!50}
	&	16	&	21-nov	&	Examen Escrito 1 (Criterio 2.1)	&	Evaluación	\\ \midrule	
9	&	17	&	26-nov	&	Razones y Proporciones: Definición y Aplicaciones	&	Aula invertida	\\	
	&	18	&	28-nov	&	Aplicaciones de Razones y Proporciones en Problemas Prácticos	&		\\ \midrule	
10	&	19	&	03-dic	&	Conceptos de Población, Muestra, Parámetros y Variables	&		\\	
	&	20	&	05-dic	&	Conceptos de Fenómenos Determinados y Aleatorios	&		\\ \midrule	
11	&	21	&	10-dic	&	Análisis de Datos Prácticos (Aplicaciones de Estadística)	&	Envío del Reto 2 (Criterio 3.3)	\\	\rowcolor{celeste!50}
	&	22	&	12-dic	&	Cuestionario en Línea 2 (Criterio 1.2)	&	Evaluación	\\ \midrule	
12	&	23	&	17-dic	&	Sistemas de Ecuaciones Lineales: Definición y Métodos	&		\\	
	&	24	&	19-dic	&	Métodos de Solución Algebraica de Sistemas de Ecuaciones	&		\\ \midrule	
	&	25	&	24-dic	&	Métodos de Solución Geométrica de Sistemas de Ecuaciones	&	Entrega del Reto 2 (Criterio 3.3)	\\	\rowcolor{verde!50}
	&		&	26-dic	&		&	Feriado	\\ \midrule	\rowcolor{verde!50}
	&		&	31-dic	&		&	Feriado	\\	
	&	26	&	02-ene	&	Inecuaciones de Dos Variables: Solución y Representación Gráfica	&	Aula invertida	\\ \midrule	\rowcolor{celeste!50}
13	&	27	&	07-ene	&	Examen Escrito 2 (Criterio 2.2)	&	Evaluación	\\	
	&	28	&	09-ene	&	Repaso y Resolución de Problemas de Ecuaciones e Inecuaciones	&		\\ \midrule	
14	&	29	&	14-ene	&	Aplicaciones de Ecuaciones e Inecuaciones	&		\\	
	&	30	&	16-ene	&	Aplicaciones de Ecuaciones e Inecuaciones	&		\\ \midrule	
15	&	31	&	21-ene	&	Aplicaciones de Ecuaciones en Contextos Reales	&		\\	
	&	32	&	23-ene	&	Aplicaciones de Inecuaciones en Contextos Reales	&		\\ \midrule	\rowcolor{celeste!50}
16	&	33	&	28-ene	&	Examen Escrito 3 (Criterio 3.2)	&	Evaluación	\\	
	&	34	&	30-ene	&	Retroalimentación	&		\\ 
\end{longtable}
\end{center}
\end{landscape}

\end{document} 