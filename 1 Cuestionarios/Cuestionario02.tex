\documentclass[a4,11pt]{aleph-notas}
% Actualizado en febrero de 2024
% Funciona con TeXLive 2022
% Para obtener solo el pdf, compilar con pdfLaTeX. 
% Para obtener el xml compilar con XeLaTeX.

% -- Paquetes adicionales 
\usepackage{aleph-moodle}
\moodleregisternewcommands
% Todos los comandos nuevos deben ir luego del comando anterior
\usepackage{aleph-comandos}


% -- Datos 
\institucion{Escuela de Ciencias Físicas y Matemática}
\carrera{Medicina Veterinaria}
\asignatura{Matemática I*}
\tema{Cuestionario en línea 2}
\autor{Andrés Merino}
\fecha{Semestre 2024-2}

\logouno[0.14\textwidth]{Logos/logoPUCE_04_ac}
\definecolor{colortext}{HTML}{0030A1}
\definecolor{colordef}{HTML}{0030A1}
\fuente{montserrat}

% -- Otros comandos



\begin{document}

\encabezado

\vspace*{-8mm}
%%%%%%%%%%%%%%%%%%%%%%%%%%%%%%%%%%%%%%%%
\section{Indicaciones}
%%%%%%%%%%%%%%%%%%%%%%%%%%%%%%%%%%%%%%%%

\begin{itemize}[leftmargin=*]
\item 
    En esta actividad se evalúa si el estudiante \textit{(Criterio 1.2) Reconoce los conceptos fundamentales de estadística, incluyendo población, muestra, parámetros y variables.}
\item
    
\end{itemize}

%%%%%%%%%%%%%%%%%%%%%%%%%%%%%%%%%%%%%%%%
\section{Banco de preguntas}
%%%%%%%%%%%%%%%%%%%%%%%%%%%%%%%%%%%%%%%%

%%%%%%%%%%%%%%%%%%%%%%%%%%%%%%%%%%%%%%%%
\begin{quiz}{Nombre del cuestionario}
%%%%%%%%%%%%%%%%%%%%%%%%%%%%%%%%%%%%%%%%

%%%%%%%%%%%%%%%%%%%%%%%%%%%%%%%%%%%%%%%%
\begin{multi}[%
    % - Retroalimentación
    feedback={La respuesta correcta es $\left[0,+\infty\right[$}
    ]%
    % - Indentificador
    {Pregunta-01}
    % - Enunciado
    ¿Cuál es la imagen de la función $\func{f}{\R}{\R}$ tal que $x\mapsto x^2$?
    \item $\R$
    \item* $\left[0,+\infty\right[$
    \item $\left]-\infty,0\right]$
\end{multi}

%%%%%%%%%%%%%%%%%%%%%%%%%%%%%%%%%%%%%%%%
\begin{numerical}[tolerance=0.1,%
    % - Retroalimentación
    feedback={Se tiene que $\sen(0)=1$.}
    ]%
    % - Indentificador
    {Pregunta-02}
    % - Enunciado
    El resultado de $\sen(0)$ es:
    \item 0
\end{numerical}

%%%%%%%%%%%%%%%%%%%%%%%%%%%%%%%%%%%%%%%%
\begin{shortanswer}[%
    % - Retroalimentación
    feedback={El dominio de la relación es $\{a,c,d\}$}
    ]%
    % - Indentificador
    {Pregunta-03}
    % - Enunciado
    Considere la relación $\{(a,b), (a,c), (d,c), (c,c)\}$, el dominio de la relación es (colocar los elementos separados por comas, en orden alfabético y sin espacios):
    \item a,c,d
\end{shortanswer}

\end{quiz}




\end{document}