\documentclass[a4,11pt]{aleph-notas}
% Actualizado en febrero de 2024
% Funciona con TeXLive 2022
% Para obtener solo el pdf, compilar con pdfLaTeX. 
% Para obtener el xml compilar con XeLaTeX.

% -- Paquetes adicionales
\usepackage{aleph-moodle}
\moodleregisternewcommands
% Todos los comandos nuevos deben ir luego del comando anterior
\usepackage{aleph-comandos}


% -- Datos 
\institucion{Escuela de Ciencias Físicas y Matemática}
\carrera{Medicina Veterinaria}
\asignatura{Matemática I*}
\tema{Cuestionario en línea 2}
\autor{Andrés Merino}
\fecha{Semestre 2024-2}

\logouno[0.14\textwidth]{Logos/logoPUCE_04_ac}
\definecolor{colortext}{HTML}{0030A1}
\definecolor{colordef}{HTML}{0030A1}
\fuente{montserrat}

% -- Otros comandos



\begin{document}

\encabezado

\vspace*{-8mm}
%%%%%%%%%%%%%%%%%%%%%%%%%%%%%%%%%%%%%%%%
\section{Indicaciones}
%%%%%%%%%%%%%%%%%%%%%%%%%%%%%%%%%%%%%%%%

\begin{itemize}[leftmargin=*]
\item 
    En esta actividad se evalúa si el estudiante \textit{(Criterio 1.2) reconoce los conceptos fundamentales de estadística, incluyendo población, muestra, parámetros y variables.}
\item
    
\end{itemize}

%%%%%%%%%%%%%%%%%%%%%%%%%%%%%%%%%%%%%%%%
\section{Banco de preguntas}
%%%%%%%%%%%%%%%%%%%%%%%%%%%%%%%%%%%%%%%%

%%%%%%%%%%%%%%%%%%%%%%%%%%%%%%%%%%%%%%%%
\begin{quiz}{Estadistica basica}
%%%%%%%%%%%%%%%%%%%%%%%%%%%%%%%%%%%%%%%%

%%%%%%%%%%%%%%%%%%%%%%%%%%%%%%%%%%%%%%%%
\begin{multi}%
    {Estadistica basica - 1}
    ¿Qué es una razón?
    \item[]* Una comparación entre dos cantidades.
    \item[] Un tipo de variable.
    \item[] Un fenómeno aleatorio.
    \item[] Ninguna de las anteriores.
\end{multi}

%%%%%%%%%%%%%%%%%%%%%%%%%%%%%%%%%%%%%%%%
\begin{multi}%
    {Estadistica basica - 2}
    ¿Cuál es la diferencia entre una muestra y una población?
    \item[] La muestra incluye todos los elementos de un grupo, mientras que la población es un subconjunto.
    \item[]* La población incluye todos los elementos de un grupo, mientras que la muestra es un subconjunto.
    \item[] Ambos términos son sinónimos.
    \item[] La muestra es siempre más grande que la población.
\end{multi}

%%%%%%%%%%%%%%%%%%%%%%%%%%%%%%%%%%%%%%%%
\begin{multi}%
    {Estadistica basica - 3}
    ¿Qué tipo de variable es la raza de un animal?
    \item[] Variable cuantitativa continua.
    \item[]* Variable cualitativa nominal.
    \item[] Variable cuantitativa discreta.
    \item[] Variable cualitativa ordinal.
\end{multi}

%%%%%%%%%%%%%%%%%%%%%%%%%%%%%%%%%%%%%%%%
\begin{multi}%
    {Estadistica basica - 4}
    ¿Cuál de los siguientes es un fenómeno determinístico?
    \item[]* El crecimiento de un animal bajo condiciones controladas.
    \item[] La aparición de enfermedades en animales por factores ambientales.
    \item[] El comportamiento aleatorio de un gato al jugar.
    \item[] Ninguna de las anteriores.
\end{multi}

%%%%%%%%%%%%%%%%%%%%%%%%%%%%%%%%%%%%%%%%
\begin{multi}%
    {Estadistica basica - 5}
    En un estudio veterinario, se observan los siguientes fenómenos:
    \begin{itemize}
    	\item  Aumento del peso en cachorros alimentados con una dieta específica.
    	\item Variación en el comportamiento de los gatos al interactuar con nuevos juguetes.
    	\item La tasa de recuperación de animales después de una cirugía.
    \end{itemize}
    ¿Cuál de estos fenómenos se clasificaría como determinístico y cuál como aleatorio?
    \item[]* El aumento del peso en cachorros es determinístico; la variación en el comportamiento de los gatos es aleatorio.
    \item[] La tasa de recuperación es determinística; el aumento del peso en cachorros es aleatorio.
    \item[] Todos son fenómenos aleatorios.
    \item[] Todos son fenómenos determinísticos.
\end{multi}

%%%%%%%%%%%%%%%%%%%%%%%%%%%%%%%%%%%%%%%%
\begin{multi}%
    {Estadistica basica - 6}
    Si en una clínica veterinaria hay 30 gatos y 45 perros, calcula la razón de gatos a perros.
    \item[]* $\frac{30}{45}$
    \item[] $\frac{45}{30}$
\end{multi}

%%%%%%%%%%%%%%%%%%%%%%%%%%%%%%%%%%%%%%%%
\begin{multi}%
    {Estadistica basica - 7}
    En una población de 200 animales, 80 son gatos y 120 son perros. ¿Cuál es la proporción de gatos en esta población?
    \item[]* $\frac{80}{200} = 0.4$
    \item[] $\frac{120}{200} = 0.6$
\end{multi}

%%%%%%%%%%%%%%%%%%%%%%%%%%%%%%%%%%%%%%%%
\begin{multi}%
    {Estadistica basica - 8}
    Si se quiere realizar un estudio sobre el comportamiento alimenticio de los perros en una ciudad con 10,000 perros, ¿cuántos perros deberías incluir en tu muestra si decides tomar el 10\%?
    \item[]* $1000$
    \item[] $100$
    \item[] $100000$
    \item[] $10$
\end{multi}

%%%%%%%%%%%%%%%%%%%%%%%%%%%%%%%%%%%%%%%%
\begin{multi}%
    {Estadistica basica - 9}
    Un veterinario ha registrado los pesos (en kg) de cinco perros que han sido atendidos en su clínica. Los pesos son los siguientes: 10, 12, 15, 8 y 14. ¿Cuál es la media del peso de estos perros?
    \item[]* $10.8$ kg
    \item[] $11.8$ kg
    \item[] $15.0$ kg
\end{multi}

%%%%%%%%%%%%%%%%%%%%%%%%%%%%%%%%%%%%%%%%
\begin{multi}%
    {Estadistica basica - 10}
    En un estudio sobre el tiempo que pasan los gatos en la clínica veterinaria, se registraron los siguientes tiempos (en minutos): 30, 45, 25, 35 y 40. ¿Cuál es la media del tiempo que pasan estos gatos en la clínica?
    \item[]* $35$ minutos
    \item[] $25$ minutos
    \item[] $45$ minutos
\end{multi}

\end{quiz}




\end{document}