\documentclass[11pt,respuestas,a4]{aleph-examen}
%\documentclass[11pt,a4]{aleph-examen}
% Se puede ver la documentación aquí: 
% https://github.com/alephsub0/LaTeX_aleph-examen

% -- Paquetes adicionales 
\usepackage{aleph-comandos}
\usepackage{booktabs}
\usepackage{multicol}
\usepackage{pgfplots}

% -- Datos 
\institucion{Escuela de Ciencias Físicas y Matemática}
\carrera{Medicina Veterinaria}
\asignatura{Matemática I*}
\tema{Examen escrito no. 2}
\autor{Fernando Jiménez T.}
\fecha{Semestre 2024-2}


\logouno[0.14\textwidth]{Logos/logoPUCE_04_ac}
\definecolor{colortext}{HTML}{0030A1}
\definecolor{colordef}{HTML}{0030A1}
\fuente{montserrat}


\begin{document}

\encabezado

\section*{Indicaciones}
\begin{itemize}[leftmargin=*]
\item 
    En esta actividad se evalúa si el estudiante \textit{(Criterio 2.2) resuelve sistemas de ecuaciones e inecuaciones de varias variables,
    	aplicando métodos algebraicos y geométricos adecuados.} 
\item
    Se encuentra prohibido el uso de cualquier fuente de información durante todo el examen.
\item
    En caso de considerar que existe un error en la pregunta o que esta se encuentra mal planteada, se debe indicar cuál es el error y justificarlo.
\item
    Todas las soluciones deben estar correctamente redactadas y explicadas.
\end{itemize}

\section*{Ejercicios}

\begin{preguntas}

%%%%%%%%%%%%%%%%%%%%%%%%%%%%%%%%%%%%%%%%
\item Complete la siguiente tabla diciendo si el sistema es lineal o no, y colocar cuántas incógnitas tiene el sistema.\puntaje{15}
\begin{center}
    \begin{tabular}{c|c|c}
        Sistema & \hspace{1cm} Tipo \hspace{1cm} & Num. Incógnitas  \\ \midrule
        $\begin{cases} 
        	2x - y = 5 \\
        	x + y^2 = 1
        \end{cases}$ & &  \\[2mm]\midrule
        $\begin{cases} 
        	2x - y = 5 \\
        	x + 2y = 1
        \end{cases}$ & &  \\[2mm]\midrule
        $\begin{cases} 
        2x - y = 5 \\
        x + y^2 + z = 1
        \end{cases}$ & &  \\[2mm]\midrule
        $\begin{cases} 
        	2x - y = 5 \\
        	x + y + z = 1
        \end{cases}$ & &  \\[2mm]\bottomrule
    \end{tabular}
\end{center}
% -----------------------------
\begin{respuesta}
\begin{center}
	\begin{tabular}{c|c|c}
		Sistema & \hspace{1cm} Tipo \hspace{1cm} & Num. Incógnitas  \\ \midrule
		$\begin{cases} 
			2x - y = 5 \\
			x + y^2 = 1
		\end{cases}$ & No lineal & 2  \\[2mm]\midrule
		$\begin{cases} 
			2x - y = 5 \\
			x + 2y = 1
		\end{cases}$ & Lineal & 2 \\[2mm]\midrule
		$\begin{cases} 
			2x - y = 5 \\
			x + y^2 + z = 1
		\end{cases}$ & No lineal & 3  \\[2mm]\midrule
		$\begin{cases} 
			2x - y = 5 \\
			x + y + z = 1
		\end{cases}$ & Lineal & 3 \\[2mm]\bottomrule
	\end{tabular}
\end{center}
\end{respuesta}
%%%%%%%%%%%%%%%%%%%%%%%%%%%%%%%%%%%%%%%%
\item Resolver los siguientes sistemas:\puntaje{20}
\begin{multicols}{2}
    \begin{enumerate}[label=\textit{\alph*)}]
        \item $\begin{cases}
        	3x - y  = 5 \\
        	2x + y = 5
        \end{cases}$
        \item $\begin{cases}
        	x + 2y  + 2z = 6 \\
        	x - y = -1 \\
        	2x + y + 3z = 7
        \end{cases}$
        \item $\begin{cases}
        2x  + 5y = 9 \\
        -x + 3y = 1 \\
        7x - 2y = 14
        \end{cases}$
        \item $\begin{cases}
        x - y  + z = 2 \\
        x + y + 3z = 6 \\
        2y + 3z = 5
        \end{cases}$
    \end{enumerate}
\end{multicols}
% -----------------------------
\begin{respuesta}
    \begin{enumerate}[label=\textit{\alph*)}]
        \item Usando el m\'etodo de eliminaci\'on podemos sumar las dos ecuaciones obteniendo la ecuaci\'on
        $$
        5x = 10
        $$
        Es decir, tenemos que $x = 2$. Reemplazando el valor de $x$ encontrado en la segunda ecuaci\'on tenemos:
        $$
        2(2) + y = 5
        $$
        Es decir, obtenemos que $y = 1$.
        % -------------
        \item Primero usaremos el m\'etodo de eliminaci\'on para obtener un sistema triangular.
        
        Multiplicamos por $-3$ a la primera ecuaci\'on y por $2$ la segunda ecuaci\'on, luego sumamos ambas ecuaciones:
        $$
        \begin{aligned}
        	-3 x - 6y - 6 z &= -18 \\
        	4x + 2y + 6z &= 14 \\ \hline 
        	x - 4y &= -4
        \end{aligned}
        $$
        Reemplazamos la tercera ecuaci\'on por esta nueva ecuaci\'on obteniendo el sistema equivalente:
        $$
        \begin{cases}
        	x + 2y  + 2z = 6 \\
        	x - y = -1 \\
        	x - 4y = -4
        \end{cases}
        $$
        Multiplicamos por $-1$ la segunda ecuaci\'on y sumamos la tercera ecuaci\'on obteniendo:
        $$
        \begin{aligned}
        	-x + y &= 1 \\
        	x - 4y &= -4 \\ \hline
        	-3y &= -3
        \end{aligned}
        $$
        Esta nueva ecuaci\'on la reemplazamos por la tercera ecuaci\'on obteniendo el sistema equivalente:
        $$
        \begin{cases}
        	x + 2y  + 2z = 6 \\
        	x - y = -1 \\
        	 -3y = -3
        \end{cases}
        $$
        Finalmente, podemos resolver por sustituci\'on, obteniendo:
        $$
        \begin{cases}
        	x = 0 \\
        	y = 1 \\
        	z = 2
        \end{cases}
        $$
        % -----------------------------
        \item Despejando $x$ de la segunda ecuaci\'on obtenemos:
        $$
        x = 3y - 1
        $$
        Reemplazando el valor de $x$ en la primera ecuaci\'on obtenemos:
        $$
        2(3y-1) + 5y = 9 \quad \Longleftrightarrow \quad y = 1
        $$
        Reemplazando el valor de $y$ encontrado en $x = 3y-1$, obtenemos
        $$
        x = 2
        $$
        Verifiquemos si estas el punto $(2,1)$ satisface la tercera ecuaci\'on:
        $$
        12 = 7(2) - 2(1) \neq 14
        $$
        Por lo tanto, el sistema no tiene soluci\'on.
        % ----------
        \item Por eliminaci\'on reduciremos el sistema a un sistema de forma triangular. 
        
        Primero multiplicamos por $-1$ la primera ecuaci\'on y el resultado sumamos con la segunda ecuaci\'on obteniendo:
        $$
        \begin{aligned}
        	-x + y - z &= -2 \\
        	x + y + 3z &= 6 \\ \hline
        	2y + 2z &= 4
        \end{aligned}
        $$
        Reemplazamos la segunda ecuaci\'on por la nueva ecuaci\'on, obteniendo el sistema equivalente:
        $$
        \begin{cases}
        	x - y  + z = 2 \\
        	2y + 2z = 4 \\
        	2y + 3z = 5
        \end{cases}
        $$
        Multiplicamos por $-1$ la segunda ecuaci\'on y sumamos con la tercera ecuaci\'on obteniendo:
        $$
        \begin{aligned}
        	-2y -2 z  &= -4 \\
        	2y + 3z &= 5 \\ \hline
        	z &= 1
        \end{aligned}
        $$
        Reemplazamos la tercera ecuaci\'on del sistema por la nueva ecuaci\'on, obteniendo:
        $$
        \begin{cases}
        	x - y  + z = 2 \\
        	2y + 2z = 4 \\
        	z = 1
        \end{cases}
        $$
        Como el sistema es triangular, podemos usar sustituci\'on para terminar de resolver, obteniendo:
        $$
        \begin{cases}
        	x = 2 \\
        	y = 1 \\
        	z = 1
        \end{cases}
        $$
    \end{enumerate}
\end{respuesta}
%%%%%%%%%%%%%%%%%%%%%%%%%%%%%
\item Una investigadora realiza un experimento para probar una hipótesis donde intervienen los nutrientes niacina y retinol. Ella alimenta a un grupo de ratas de laboratorio con una dieta diaria de precisamente 32 unidades de niacina y 22 mil unidades de retinol. Ella usa dos tipos de alimentos comerciales en forma de pastillas. El alimento A contiene 0.12 unidades de niacina y 100 unidades de retinol por gramo; el alimento B contiene 0.20 unidades de niacina y 50 unidades de retinol por gramo.

¿Cuántos gramos de cada alimento les da ella al grupo de ratas diariamente? \puntaje{15}
% -----------------------------
\begin{respuesta}
	Definamos las variables:
	\begin{itemize}
		\item[] $x$ es la cantidad en gramos del alimento $A$.
		\item[] $y$ es la cantidad en gramos del alimento $B$.
	\end{itemize}
	La primera ecuaci\'on nos dice la cantidad total de niacina que se obtuvo al mezclar ambos alimentos, es decir,
	$$
	0.12 x + 0.20 y = 32
	$$
	La segunda ecuaci\'on nos dice la cantidad total de retinol que se obtiene al mezclar los alimentos.
	$$
	100 x + 50 y = 22 000
	$$
	Es decir, el sistema que debemos resolver es:
	$$
	\begin{cases}
		0.12 x + 0.20 y = 32 \\
		100 x + 50 y = 22 000
	\end{cases}
	$$
	Un sistema equivalente se obtiene al multiplicar por 100 a la primera restricci\'on, obteniendo:
	$$
	\begin{cases}
		12 x + 20 y = 3200 \\
		100 x + 50 y = 22 000
	\end{cases}
	$$
	Despejando $y$ de la primera restricci\'on obtenemos:
	$$
	y = \frac{3200 - 12x }{20}
	$$
	Reemplazando el valor de $y$ en la segunda restricci\'on tenemos:
	$$
	100 x + 50 \left( \frac{3200 - 12x }{20} \right) = 22000
	$$
	Resolviendo esta ecuaci\'on lineal tenemos:
	$$
	x = 200
	$$
	Reemplazando el valor de $x = 200$ en la la ecuaci\'on $y = \frac{3200 - 12x}{20}$, tenemos
	$$
	y = 40
	$$
	La investigadora ha usado 200 gramos del alimento A y 40 gramos del alimento B para alimentar a las ratas.
\end{respuesta}
% -----------------------------
\end{preguntas}
\end{document}