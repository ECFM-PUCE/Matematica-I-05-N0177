\documentclass[a4,11pt]{aleph-notas}
% Se puede ver la documentación aquí: 
% https://github.com/alephsub0/LaTeX_aleph-notas

% -- Paquetes adicionales 
\usepackage{enumitem}
\usepackage{aleph-comandos}
\usepackage{booktabs}


% -- Datos 
\institucion{Escuela de Ciencias Físicas y Matemática}
\carrera{Medicina Veterinaria}
\asignatura{Matemática I*}
\tema{Reto no. 2}
\autor{Fernando Jiménez T.}
\fecha{Semestre 2024-2}

\logouno[0.14\textwidth]{Logos/logoPUCE_04_ac}
\definecolor{colortext}{HTML}{0030A1}
\definecolor{colordef}{HTML}{0030A1}
\fuente{montserrat}


% -- Comandos adicionales
\setlist[enumerate]{label=\roman*.}


\begin{document}

\encabezado

\vspace*{-10mm}
%%%%%%%%%%%%%%%%%%%%%%%%%%%%%%%%%%%%%%%%
\section*{Indicaciones}
\begin{itemize}[leftmargin=*]
\item 
    En esta actividad se evalúa si el estudiante \textit{(Criterio 3.3) utiliza conceptos estadísticos básicos para el análisis de datos relevantes en la práctica, como tasas de crecimiento, dosificación de medicamentos, análisis de poblaciones, etc.}
\item 
	El resultado de aprendizaje al que apunta esta actividad es (RdA3): Aplicar los conceptos de matemática básica y estadística básica en la resolución de problemas prácticos en diversos contextos.
\item 
	El trabajo debe ser resuelto en grupos de 4 estudiantes.
\end{itemize}

%%%%%%%%%%%%%%%%%%%%%%%%%%%%%%%%%%%%%%%%
\section{Objetivos}
%%%%%%%%%%%%%%%%%%%%%%%%%%%%%%%%%%%%%%%%
El propósito de este reto es que los estudiantes de veterinaria apliquen conceptos básicos de estadística para analizar datos relevantes en su campo, como tasas de crecimiento, dosificación de medicamentos y análisis de poblaciones. Este ejercicio les permitirá familiarizarse con la recolección y el análisis de datos, así como con la interpretación de resultados.

%%%%%%%%%%%%%%%%%%%%%%%%%%%%%%%%%%%%%%%%
\section{Descripción}
%%%%%%%%%%%%%%%%%%%%%%%%%%%%%%%%%%%%%%%%
Los estudiantes simularán trabajar en una clínica veterinaria donde deberán recolectar y analizar datos sobre un grupo de animales. A continuación, se presentan las instrucciones para llevar a cabo el reto.

\subsection{Definición de la Población y Muestra:}
\begin{itemize}
	\item Los estudiantes deberán definir una población de animales (por ejemplo, gatos o perros).
	\item Deberán seleccionar una muestra aleatoria de al menos 20 animales para realizar el análisis. Para esta selecci\'on, se puede escoger 20 estudiantes de la universidad (al azar) y realizarles diferentes preguntas para recolectar informaci\'on sobre sus mascotas.
\end{itemize}
% ------------------
\subsection{Recolecci\'on de Datos:}
Los estudiantes deben recolectar los siguientes datos para cada mascota:
	\begin{itemize}
		\item Sexo: Masculino o Femenino
		\item Edad: En años humanos.
		\item Peso al nacer: En kilogramos.
		\item Peso actual: En kilogramos.
		\item Raza: Clasificaci\'on est\'andar (por ejemplo, pura, mestizo, etc.)
		\item Estado de salud: Sano o enfermo.
	\end{itemize}
% ------------------
\subsection{An\'alisis estad\'istico:}
Calcular las siguientes medidas descriptivas para cada uno de los par\'ametros definidos anteriormente. En el caso de que no sea posible calcular alguna de las medidas, explicar por qu\'e.
	\begin{itemize}
	\item Media
	\item Mediana 
	\item Moda
	\item Frecuencias
	\item Tasa de crecimiento promedio
\end{itemize}
% ------------------
\subsection{Interpretaci\'on de resultados:}
Usando las medidas calculadas, realizar al menos dos conclusiones, interpretando los resultados obtenidos y discutiendo posibles implicaciones para la salud de las mascotas.
%%%%%%%%%%%%%%%%%%%%%%%%%%%%%%%%%%%%%%%%
\section{Producto}
%%%%%%%%%%%%%%%%%%%%%%%%%%%%%%%%%%%%%%%%
El producto final incluirá un documento explicativo y un video tutorial, donde se indique:
\begin{itemize}
	\item La población seleccionada.
	\item ¿Cómo se seleccionó la muestra? Un breve resumen del proceso de recolección de datos.
	\item ¿Cómo se calcularon las diferentes medidas?
	\item Las conclusiones.
\end{itemize}
%%%%%%%%%%%%%%%%%%%%%%%%%%%%%%%%%%%%%%%%
\section{Rúbrica de evaluación}
%%%%%%%%%%%%%%%%%%%%%%%%%%%%%%%%%%%%%%%%
\begin{itemize}
	\item \textbf{Definiciones (10 pts):} Se evaluar\'a la claridad al seleccionar la poblaci\'on de estudio y su importancia.
	\item \textbf{Recolecci\'on de datos (15 pts):} Se evaluar\'a la precisi\'on al recolectar y organizar los datos.
	\item \textbf{An\'alisis estad\'istico (15 pts):} Se evaluar\'a la precisi\'on en los c\'alculos estad\'isticos.
	\item \textbf{Conclusiones (10 pts):} Se evaluar\'a la capacidad para interpretar los resultados y hacer recomendaciones basadas en el an\'alisis estad\'istico.
\end{itemize}
\end{document}