% \documentclass[11pt,respuestas,a4]{aleph-examen}
\documentclass[11pt,a4]{aleph-examen}
% Se puede ver la documentación aquí: 
% https://github.com/alephsub0/LaTeX_aleph-examen

% -- Paquetes adicionales
\usepackage{aleph-comandos}
\usepackage{booktabs}
\usepackage{multicol}

% -- Datos 
\institucion{Escuela de Ciencias Físicas y Matemática}
\carrera{Medicina Veterinaria}
\asignatura{Matemática I*}
\tema{Taller no. 1}
\autor{Andrés Merino}
\fecha{Semestre 2024-2}


\logouno[0.14\textwidth]{Logos/logoPUCE_04_ac}
\definecolor{colortext}{HTML}{0030A1}
\definecolor{colordef}{HTML}{0030A1}
\fuente{montserrat}


\begin{document}

\encabezado

\section*{Indicaciones}
\begin{itemize}[leftmargin=*]
\item 
    En esta actividad se evalúa si el estudiante \textit{(Criterio 1.1) comprende los conceptos fundamentales de lógica matemática, teoría de conjuntos y números reales aplicables en su campo.}
\item 
    El taller se desarrollará de manera colaborativa en grupos de trabajo asignados por el docente.
\item 
    Se permitirá el uso de libros de texto y notas de clase.
\item 
    Cada grupo deberá resolver los problemas propuestos y justificar cada solución de manera clara, indicando los procedimientos utilizados y los conceptos aplicados.
\item 
    Al final del taller, cada grupo presentará sus resultados y explicará sus razonamientos al resto de la clase, mediante un video.
\end{itemize}


\section*{Ejercicios}

\begin{preguntas}

%%%%%%%%%%%%%%%%%%%%%%%%%%%%%%%%%%%%%%%%
%%%%%%%%%%%%%%%%%%%%%%%%%%%%%%%%%%%%%%%%
%%%%%%%%%%%%%%%%%%%%%%%%%%%%%%%%%%%%%%%%
\item
    Pregunta

\begin{respuesta}
    Solución
\end{respuesta}


\end{preguntas}


\end{document}