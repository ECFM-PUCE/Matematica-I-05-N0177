\documentclass[a4,11pt]{aleph-notas}
% Se puede ver la documentación aquí: 
% https://github.com/alephsub0/LaTeX_aleph-notas


% -- Paquetes adicionales 
\usepackage{enumitem}
\usepackage{aleph-comandos}
\usepackage{multicol}

% -- Datos 
\institucion{Escuela de Ciencias Físicas y Matemática}
\carrera{Medicina Veterinaria}
\asignatura{Matemática I*}
\tema{Ejercicios no. 1: Conceptos básicos de lógica}
\autor{Andrés Merino}
\fecha{Semestre 2024-2}


\logouno[0.14\textwidth]{Logos/logoPUCE_04_ac}
\definecolor{colortext}{HTML}{0030A1}
\definecolor{colordef}{HTML}{0030A1}
\fuente{montserrat}

% -- Comandos adicionales
\setlist[enumerate]{label=\roman*.}

\begin{document}

\encabezado

%%%%%%%%%%%%%%%%%%%%%%%%%%%%%%%%%%%%%%%%
%%%%%%%%%%%%%%%%%%%%%%%%%%%%%%%%%%%%%%%%
%%%%%%%%%%%%%%%%%%%%%%%%%%%%%%%%%%%%%%%%
\begin{ejer}
    Determinar si las siguientes proposiciones son verdaderas o falsas:
    \begin{multicols}{2}
    \begin{enumerate}[label=\textit{\alph*)}]
        \item $2 + 3 = 5$
        \item $4 > 7$
        \item $\text{El cielo es azul.}$
        \item $\pi$ es un número irracional.
    \end{enumerate}
    \end{multicols}
\end{ejer}

\begin{proof}[Solución]\hspace{0pt}
    
\end{proof}

%%%%%%%%%%%%%%%%%%%%%%%%%%%%%%%%%%%%%%%%
%%%%%%%%%%%%%%%%%%%%%%%%%%%%%%%%%%%%%%%%
%%%%%%%%%%%%%%%%%%%%%%%%%%%%%%%%%%%%%%%%
\begin{ejer}
    Escribir las siguientes expresiones en lenguaje formal usando letras proposicionales y conectores lógicos:
    \begin{enumerate}[label=\textit{\alph*)}]
        \item Si está lloviendo, entonces llevo paraguas.
        \item O es lunes, o es martes.
        \item No es cierto que $3 > 5$.
        \item Si $x > 0$, entonces $x^2 > 0$.
    \end{enumerate}
\end{ejer}

\begin{proof}[Solución]\hspace{0pt}
    
\end{proof}

%%%%%%%%%%%%%%%%%%%%%%%%%%%%%%%%%%%%%%%%
%%%%%%%%%%%%%%%%%%%%%%%%%%%%%%%%%%%%%%%%
%%%%%%%%%%%%%%%%%%%%%%%%%%%%%%%%%%%%%%%%
\begin{ejer}
    Construir la tabla de verdad para las siguientes proposiciones:
    \begin{multicols}{2}
    \begin{enumerate}[label=\textit{\alph*)}]
        \item $\neg p \land q$
        \item $(p \land q) \lor q$
        \item $(p \lor q) \land q$
        \item $p \rightarrow (q \lor p)$
    \end{enumerate}
    \end{multicols}
    \vspace{0pt}
\end{ejer}

\begin{proof}[Solución]\hspace{0pt}
    
\end{proof}

%%%%%%%%%%%%%%%%%%%%%%%%%%%%%%%%%%%%%%%%
%%%%%%%%%%%%%%%%%%%%%%%%%%%%%%%%%%%%%%%%
%%%%%%%%%%%%%%%%%%%%%%%%%%%%%%%%%%%%%%%%
\begin{ejer}
    Determinar si las siguientes proposiciones son tautologías, contradicciones o contingencias:
    \begin{multicols}{2}
    \begin{enumerate}[label=\textit{\alph*)}]
        \item $p \vee \neg p$
        \item $p \wedge \neg p$
        \item $p \rightarrow p$
        \item $p \rightarrow \neg p$
    \end{enumerate}
    \end{multicols}
    \vspace{0pt}
\end{ejer}

\begin{proof}[Solución]\hspace{0pt}
    
\end{proof}

%%%%%%%%%%%%%%%%%%%%%%%%%%%%%%%%%%%%%%%%
%%%%%%%%%%%%%%%%%%%%%%%%%%%%%%%%%%%%%%%%
%%%%%%%%%%%%%%%%%%%%%%%%%%%%%%%%%%%%%%%%
\begin{ejer}
    Dadas las proposiciones $p$: «Hoy es martes», y $q$: «Está soleado», escribir las siguientes expresiones en lenguaje natural:
    \begin{multicols}{2}
    \begin{enumerate}[label=\textit{\alph*)}]
        \item $p \wedge q$
        \item $p \vee q$
        \item $\neg p \rightarrow q$
        \item $p \wedge \neg q$
    \end{enumerate}
    \end{multicols}
    \vspace{0pt}
\end{ejer}

\begin{proof}[Solución]\hspace{0pt}
    
\end{proof}

\end{document}