\documentclass[a4,11pt]{aleph-notas}
% Se puede ver la documentación aquí: 
% https://github.com/alephsub0/LaTeX_aleph-notas


% -- Paquetes adicionales 
\usepackage{enumitem}
\usepackage{aleph-comandos}
\usepackage{multicol}

% -- Datos 
\institucion{Escuela de Ciencias Físicas y Matemática}
\carrera{Medicina Veterinaria}
\asignatura{Matemática I*}
\tema{Ejercicios no. 2: Conceptos básicos de teoría de conjuntos}
\autor{Andrés Merino}
\fecha{Semestre 2024-2}


\logouno[0.14\textwidth]{Logos/logoPUCE_04_ac}
\definecolor{colortext}{HTML}{0030A1}
\definecolor{colordef}{HTML}{0030A1}
\fuente{montserrat}

% -- Comandos adicionales
\setlist[enumerate]{label=\roman*.}

\begin{document}

\encabezado

%%%%%%%%%%%%%%%%%%%%%%%%%%%%%%%%%%%%%%%%
%%%%%%%%%%%%%%%%%%%%%%%%%%%%%%%%%%%%%%%%
%%%%%%%%%%%%%%%%%%%%%%%%%%%%%%%%%%%%%%%%
\begin{ejer}
    Dados los conjuntos $A = \{1, 2, 3, 4\}$ y $B = \{2, 4, 6, 8\}$, determina si $A$ es un subconjunto de $B$ y si $B$ es un subconjunto de $A$.
\end{ejer}

\begin{proof}[Solución]\hspace{0pt}
    
\end{proof}

%%%%%%%%%%%%%%%%%%%%%%%%%%%%%%%%%%%%%%%%
%%%%%%%%%%%%%%%%%%%%%%%%%%%%%%%%%%%%%%%%
%%%%%%%%%%%%%%%%%%%%%%%%%%%%%%%%%%%%%%%%
\begin{ejer}
    Sean $A = \{x \in \mathbb{Z} : x \geq 0 \text{ y } x \leq 10\}$ y $B = \{x \in \mathbb{Z} : x \geq 5 \text{ y } x \leq 15\}$.
    \begin{itemize}
        \item Calcula $A \cup B$.
        \item Calcula $A \cap B$.
    \end{itemize}
\end{ejer}

\begin{proof}[Solución]\hspace{0pt}
    
\end{proof}

%%%%%%%%%%%%%%%%%%%%%%%%%%%%%%%%%%%%%%%%
%%%%%%%%%%%%%%%%%%%%%%%%%%%%%%%%%%%%%%%%
%%%%%%%%%%%%%%%%%%%%%%%%%%%%%%%%%%%%%%%%
\begin{ejer}
    Dados $A = \{a, b, c, d\}$ y $B = \{b, d, e, f\}$, determina $A \setminus B$ y $B \setminus A$.
\end{ejer}

\begin{proof}[Solución]\hspace{0pt}
    
\end{proof}

%%%%%%%%%%%%%%%%%%%%%%%%%%%%%%%%%%%%%%%%
%%%%%%%%%%%%%%%%%%%%%%%%%%%%%%%%%%%%%%%%
%%%%%%%%%%%%%%%%%%%%%%%%%%%%%%%%%%%%%%%%
\begin{ejer}
    Si el conjunto universal es $U = \{a, b, c, d, e, f, g\}$ y $A = \{a, c, e\}$, encuentra el complemento $A^c$.
\end{ejer}

\begin{proof}[Solución]\hspace{0pt}
    
\end{proof}

%%%%%%%%%%%%%%%%%%%%%%%%%%%%%%%%%%%%%%%%
%%%%%%%%%%%%%%%%%%%%%%%%%%%%%%%%%%%%%%%%
%%%%%%%%%%%%%%%%%%%%%%%%%%%%%%%%%%%%%%%%
\begin{ejer}
    Dibuja la representación gráfica de los siguientes intervalos:
    \begin{itemize}
        \item 
        $\left]-\infty, 3\right[$
        \item 
        $\left]-\infty, 3\right]$
        \item 
        $\left[2, 8\right[$
        \item 
        $\left]0, 1\right] \cup \left]1, 2\right]$
    \end{itemize}
\end{ejer}

\begin{proof}[Solución]\hspace{0pt}
    
\end{proof}

%%%%%%%%%%%%%%%%%%%%%%%%%%%%%%%%%%%%%%%%
%%%%%%%%%%%%%%%%%%%%%%%%%%%%%%%%%%%%%%%%
%%%%%%%%%%%%%%%%%%%%%%%%%%%%%%%%%%%%%%%%
\begin{ejer}
    Escribe los intervalos en notación de conjuntos:
    \begin{itemize}
        \item 
        $\left]-\infty, 5\right] \cup \left[4, 12\right[$
        \item 
        $\left[2, 6\right[ \cap \left]5, 10\right]$
        \item 
        $\left]-3, 3\right] \setminus \left]-1, 4\right[$
    \end{itemize}
\end{ejer}

\begin{proof}[Solución]\hspace{0pt}
    
\end{proof}

%%%%%%%%%%%%%%%%%%%%%%%%%%%%%%%%%%%%%%%%
%%%%%%%%%%%%%%%%%%%%%%%%%%%%%%%%%%%%%%%%
%%%%%%%%%%%%%%%%%%%%%%%%%%%%%%%%%%%%%%%%
\begin{ejer}
    Un grupo de 30 estudiantes fue encuestado sobre sus preferencias de películas. Se encontró que 18 estudiantes disfrutan de películas de acción, 12 disfrutan de películas de comedia, y 5 disfrutan de ambos géneros. ¿Cuántos estudiantes no disfrutan de ninguno de estos géneros?
\end{ejer}

\begin{proof}[Solución]\hspace{0pt}
    
\end{proof}



\end{document}