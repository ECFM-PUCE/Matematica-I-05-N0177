\documentclass[a4,11pt]{aleph-notas}
% Se puede ver la documentación aquí: 
% https://github.com/alephsub0/LaTeX_aleph-notas


% -- Paquetes adicionales 
\usepackage{enumitem}
\usepackage{aleph-comandos}
\usepackage{multicol}

% -- Datos 
\institucion{Escuela de Ciencias Físicas y Matemática}
\carrera{Veterinaria}
\asignatura{Matemática I*}
\tema{Ejercicios no. 4: Expresiones algebraicas}
\autor{Fernando Jiménez T.}
\fecha{Semestre 2024-2}


\logouno[0.14\textwidth]{Logos/logoPUCE_04_ac}
\definecolor{colortext}{HTML}{0030A1}
\definecolor{colordef}{HTML}{0030A1}
\fuente{montserrat}

% -- Comandos adicionales
\setlist[enumerate]{label=\roman*.}

\begin{document}

\encabezado

%%%%%%%%%%%%%%%%%%%%%%%%%%%%%%%%%%%%%%%%
\begin{ejer}
	Complete la siguiente tabla diciendo si el polinomio es un monomio, binomio o trinomio. Adem\'as, mencione cuantos t\'erminos tiene y el grado que tiene.
	\begin{center}
		\begin{tabular}{c|c|c|c}
			Polinomio & Tipo & T\'erminos & Grado \\ \hline
			$x^2 - 3x + 7$ &&&  \\
			$2x^5 + 4x^2$ &&& \\
			$-5$ &&& \\
			$x^7$ &&&  \\
			$x-x^2+x^3-x^4$ &&& \\
			$\sqrt{2} x - \sqrt{3}$ &&&
		\end{tabular}
	\end{center}
\end{ejer}
%%%%%%%%%%%%%%%%%%%%%%%%%%%%%%%%%%%%%%%%
%%%%%%%%%%%%%%%%%%%%%%%%%%%%%%%%%%%%%%%%
\begin{ejer}
    Dados los polinomios
    $$
    p(x) = x^3-6x^2 + 2x + 4 \quad \text{y} \quad q(x) = x^3 + 5x^2 - 7x
    $$
    \begin{enumerate}[label=\textit{\alph*)}]
        \item Encontrar la suma de $p(x)$ y $q(x)$
        \item Encontrar la resta de $p(x)$ y $q(x)$
    \end{enumerate}
\end{ejer}

\begin{proof}[Solución]\hspace{0pt}
   
\end{proof}

%%%%%%%%%%%%%%%%%%%%%%%%%%%%%%%%%%%%%%%%

\begin{ejer}
	Dados los polinomios
	$$
	p(x) = 2x+1  \quad \text{y} \quad q(x) =3x-5
	$$
	\begin{enumerate}[label=\textit{\alph*)}]
		\item Encontrar el producto de $p(x)$ con $q(x)$
		\item Encontrar el producto de $p(x)$ con $p(x)$
		\item Encontrar el producto de $q(x)$ con $q(x)$
	\end{enumerate}
\end{ejer}
\begin{proof}[Solución]\hspace{0pt}
	
\end{proof}
%%%%%%%%%%%%%%%%%%%%%%%%%%%%%%%%%%%%%%%%

\begin{ejer}
	Usando las f\'ormulas de productos resolver:
	\begin{multicols}{2}
	\begin{enumerate}[label=\textit{\alph*)}]
		\item $(3x+5)^2$
		\item $(x^2 - 2)^3$
		\item $(x+y-1)(x+y + 1)$
		\item $(2x - \sqrt{y})(2x + \sqrt{y})$
	\end{enumerate}
	\end{multicols}
\end{ejer}

\begin{proof}[Solución]\hspace{0pt}
    
\end{proof}

%%%%%%%%%%%%%%%%%%%%%%%%%%%%%%%%%%%%%%%%
%%%%%%%%%%%%%%%%%%%%%%%%%%%%%%%%%%%%%%%%
%%%%%%%%%%%%%%%%%%%%%%%%%%%%%%%%%%%%%%%%
\begin{ejer}
    Factorizar las siguientes expresiones:
    \begin{multicols}{2}
    \begin{enumerate}[label=\textit{\alph*)}]
        \item $3x^2 - 6x$
        \item $8x^4y^2+6x^3y^3-2xy^4$
        \item $(2x+4)(x-3)-5(x-3)$
        \item $x^2+7x+12$
        \item $6x^2 + 7x - 5$
        \item $(5x+1)^2 - 2(5x+1) - 3$
    \end{enumerate}
    \end{multicols}
    \vspace{0pt}
\end{ejer}

\begin{proof}[Solución]\hspace{0pt}
    
\end{proof}

%%%%%%%%%%%%%%%%%%%%%%%%%%%%%%%%%%%%%%%%
%%%%%%%%%%%%%%%%%%%%%%%%%%%%%%%%%%%%%%%%

\begin{ejer}
    Usando las f\'ormulas especiales factorizar:
    \begin{multicols}{2}
    \begin{enumerate}[label=\textit{\alph*)}]
        \item $4x^2 - 25$
        \item $(x+y)^2 - z^2$
        \item $27x^3 - 1$
        \item $x^6 + 8$
        \item $x^3 + x^2 + 4x + 4$
        \item $x^3 - 2x^2 - 3x + 6$
    \end{enumerate}
    \end{multicols}
    \vspace{0pt}
\end{ejer}

\begin{proof}[Solución]\hspace{0pt}
    
\end{proof}

%%%%%%%%%%%%%%%%%%%%%%%%%%%%%%%%%%%%%%%%
\begin{ejer}
	Hallar el domino de las siguientes expresiones:
	\begin{enumerate}[label=\textit{\alph*)}]
		\item $2x^2 + 3x -1$
		\item $\frac{x}{x^2 - 5x + 6}$
		\item $\frac{\sqrt{x}}{x-5}$
	\end{enumerate}
\end{ejer}

\begin{proof}[Solución]\hspace{0pt}
	
\end{proof}

%%%%%%%%%%%%%%%%%%%%%%%%%%%%%%%%%%%%%%%%
\begin{ejer}
	Simplificar la siguiente expresi\'on 
	$$
	\frac{x^2 - 1}{x^2 + x - 2}
	$$
\end{ejer}
\begin{proof}[Solución]\hspace{0pt}
	
\end{proof}
\end{document}