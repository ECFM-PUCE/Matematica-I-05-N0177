\documentclass[a4,11pt]{aleph-notas}
% Se puede ver la documentación aquí: 
% https://github.com/alephsub0/LaTeX_aleph-notas


% -- Paquetes adicionales 
\usepackage{enumitem}
\usepackage{aleph-comandos}
\usepackage{multicol}

% -- Datos 
\institucion{Escuela de Ciencias Físicas y Matemática}
\carrera{Veterinaria}
\asignatura{Matemática I*}
\tema{Resumen no. 5: Ecuaciones lineales y cuadráticas}
\autor{Andrés Merino}
\fecha{Semestre 2024-2}


\logouno[0.14\textwidth]{Logos/logoPUCE_04_ac}
\definecolor{colortext}{HTML}{0030A1}
\definecolor{colordef}{HTML}{0030A1}
\fuente{montserrat}

% -- Comandos adicionales
\setlist[enumerate]{label=\roman*.}

\begin{document}

\encabezado

%%%%%%%%%%%%%%%%%%%%%%%%%%%%%%%%%%%%%%
\section{Ecuaciones lineales}
%%%%%%%%%%%%%%%%%%%%%%%%%%%%%%%%%%%%%%

\begin{ejer}
    Resolver la ecuación $2x+1=5$.
\end{ejer}

\begin{ejer}
    Resolver la ecuación $7x – 4 = 3x + 8$.
\end{ejer}

\begin{ejer}
    Despeje \(M\) de la ecuación siguiente.
    \[
        F = G \frac{mM}{r^2}
    \]
\end{ejer}

\begin{ejer}
    Resolver la ecuación $\dfrac{3}{x+1} = 2$.
\end{ejer}

\begin{advertencia}
    Para más ejercicios, revisar la sección 1.5 (pág. 55) del libro: Stewart, J., Redlin, L. y Watson, S. (2017). \textit{Precálculo: matemáticas para el cálculo} (7.ª ed.). Cengage Learning.
\end{advertencia}



%%%%%%%%%%%%%%%%%%%%%%%%%%%%%%%%%%%%%%
\section{Ecuaciones cuadráticas}
%%%%%%%%%%%%%%%%%%%%%%%%%%%%%%%%%%%%%%

\begin{ejer}
    Resolver la ecuación $x^2 - 25 = 0$.
\end{ejer}

\begin{ejer}
    Resolver la ecuación $x^2 + 5x = 24$.
\end{ejer}

\begin{ejer}
    Resolver la ecuación $\dfrac{3}{x} + \dfrac{5}{x+2} = 2.$.
\end{ejer}

\begin{ejer}
    Resolver la ecuación $\dfrac{x^2-4}{x+2} = 1$.
\end{ejer}

\begin{ejer}
    La población de peces de cierto lago sube y baja de acuerdo con la fórmula:
    \[
        F = 1000(30 + 17t - t^2)
    \]
    Aquí \(F\) es el número de peces en el tiempo \(t\), donde \(t\) se mide en años desde el 1 de enero de 2002, cuando la población de peces se estimó por primera vez.
    
    \begin{enumerate}
        \item ¿En qué fecha la población de peces será otra vez la misma de como era el 1 de enero de 2002?
        \item ¿Antes de qué fecha habrán muerto todos los peces del lago?
    \end{enumerate}
\end{ejer}


\begin{advertencia}
    Para más ejercicios, revisar la sección 1.5 (pág. 55) del libro: Stewart, J., Redlin, L. y Watson, S. (2017). \textit{Precálculo: matemáticas para el cálculo} (7.ª ed.). Cengage Learning.
\end{advertencia}


 
\end{document}