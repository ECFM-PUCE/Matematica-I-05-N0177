\documentclass[a4,11pt]{aleph-notas}
% Se puede ver la documentación aquí: 
% https://github.com/alephsub0/LaTeX_aleph-notas

% -- Paquetes adicionales 
\usepackage{enumitem}
\usepackage{aleph-comandos}
\usepackage{booktabs}


% -- Datos 
\institucion{Escuela de Ciencias Físicas y Matemática}
\carrera{Medicina Veterinaria}
\asignatura{Matemática I*}
\tema{Resumen no. 1: Conceptos básicos de lógica}
\autor{Andrés Merino}
\fecha{Semestre 2024-2}

\logouno[0.14\textwidth]{Logos/logoPUCE_04_ac}
\definecolor{colortext}{HTML}{0030A1}
\definecolor{colordef}{HTML}{0030A1}
\fuente{montserrat}


% -- Comandos adicionales
\setlist[enumerate]{label=\roman*.}


\begin{document}

\encabezado

%%%%%%%%%%%%%%%%%%%%%%%%%%%%%%%%%%%%%%%%
\section{Introducción a la lógica}
%%%%%%%%%%%%%%%%%%%%%%%%%%%%%%%%%%%%%%

\begin{defi}[Funciones del lenguaje]
    Las funciones del lenguaje se clasifican en tres categorías principales:
    
    \begin{itemize}
        \item \textbf{Función informativa}: Esta función se manifiesta cuando el emisor transmite información objetiva acerca de la realidad. Las afirmaciones pueden ser evaluadas como verdaderas o falsas, dependiendo de su correspondencia con los hechos.
        
        \item \textbf{Función emotiva}: En esta función, el emisor comunica sus emociones, sentimientos o estados anímicos, proporcionando una visión subjetiva de su experiencia personal.
        
        \item \textbf{Función directiva}: Su propósito es incidir en la conducta del receptor, buscando orientar o motivar una acción específica por parte de este.
    \end{itemize}
\end{defi}



\begin{defi}[Proposición]
    Una proposición es una frase o enunciado que tiene una función informativa, es decir, afirma o niega algo que puede ser evaluado como verdadero o falso. Las proposiciones pueden clasificarse en dos tipos: simples o compuestas.
    
    \begin{itemize}
        \item \textbf{Proposiciones simples}: Se representan mediante una única letra, conocida como \emph{letra proposicional}.
        \item \textbf{Proposiciones compuestas}: Se forman a partir de varias letras proposicionales, combinadas mediante conectores lógicos.
    \end{itemize}
\end{defi}



\begin{defi}[Conectores lógicos]
     Los conectores lógicos son símbolos que se utilizan para formar proposiciones compuestas a partir de proposiciones simples. Los conectores que usaremos en este curso son los siguientes:
    
    \begin{itemize}[leftmargin=*]
    \item
        \textbf{Conjunción:} Expresa la idea de ``y'' y se simboliza por:
        \[
            \wedge\texto \&\&
        \]
    \item
        \textbf{Negación:} Expresa la idea de ``no'', y se simboliza por:
        \[
            \neg\texto !
        \]
    \item
        \textbf{Disyunción:} Expresa la idea de ``o'' y se lo simboliza por:
        \[
            \vee\texto || 
        \]
    \item
        \textbf{Implicación:} Expresa la idea de enunciados del tipo  ``si..., entonces''. Se simboliza por:

        \[
            \Longrightarrow\texto ->
        \]
    \item
        \textbf{Doble implicación:} Expresa la idea de enunciados del tipo ``... si y solo si...'' o ``... siempre y cuando...". Se simboliza por:
        \[
            \Longleftrightarrow\texto <->
        \]
    \end{itemize}
\end{defi}

\begin{defi}[Valor de verdad]
    El valor de verdad es la clasificación de un enunciado como verdadero o falso, dependiendo de la asignación de valores de verdad a las letras proposicionales que lo componen.
\end{defi}


\begin{defi}[Tablas de verdad]
    Las tablas de verdad nos permiten determinar el valor de verdad de proposiciones compuestas a partir de los valores de verdad de sus proposiciones simples. A continuación, se describen los comportamientos de los principales conectores lógicos:
    
    \begin{itemize}[leftmargin=*]
        \item \textbf{Negación:} Dada una proposición, si su valor de verdad es verdadero, el valor de verdad de su negación es falso; y si su valor de verdad es falso, el valor de verdad de su negación es verdadero.
        
        \item \textbf{Conjunción:} Dadas dos proposiciones, el valor de verdad de la conjunción es verdadero únicamente cuando ambas proposiciones son verdaderas.
        
        \item \textbf{Disyunción:} Dadas dos proposiciones, el valor de verdad de su disyunción es verdadero en todos los casos, excepto cuando ambas proposiciones son falsas.
                
        \item \textbf{Implicación:} Dadas dos proposiciones, el valor de verdad de la implicación es falso únicamente cuando la primera proposición es verdadera y la segunda es falsa.
        
        \item \textbf{Doble implicación:} Dadas dos proposiciones, el valor de verdad de la doble implicación es verdadero únicamente cuando ambas proposiciones tienen el mismo valor de verdad.
    \end{itemize}
    
    \begin{center}
    \begin{tabular}{ccccccc} \toprule
       & & \rotatebox{90}{Negación} & \rotatebox{90}{Disyunción} &\rotatebox{90}{Conjunción}&\rotatebox{90}{Implicación}&\rotatebox{90}{Doble implicación}  \\
    \midrule
       $p$  & $q$ & $\neg p$  & $p \vee q$ & $p \wedge q$ & $p \rightarrow q$ & $p \leftrightarrow q$ \\\midrule
        $V$ & $V$ & $F$ & $V$ & $V$ & $V$ & $V$\\
        $V$ & $F$ & $F$ & $V$ & $F$ & $F$ & $F$\\
        $F$ & $V$ & $V$ & $V$ & $F$ & $V$ & $F$\\
        $F$ & $F$ & $V$ & $F$ & $F$ & $V$ & $V$\\ \bottomrule
    \end{tabular}
    \end{center}
\end{defi}

\begin{defi}[Tautología]
    Una proposición compuesta es una \emph{tautología} si su valor de verdad es siempre verdadero, independientemente de los valores de verdad de sus componentes.
\end{defi}

\begin{defi}[Contradicción]
    Una proposición compuesta es una \emph{contradicción} si su valor de verdad es siempre falso, sin importar los valores de verdad de sus componentes. Si una proposición compuesta no es ni tautología ni contradicción, se dice que es una \emph{contingencia}.
\end{defi}

\begin{defi}[Contingencia]
    Si una proposición compuesta no es ni tautología ni contradicción, se dice que es una \emph{contingencia}.
\end{defi}




\end{document}