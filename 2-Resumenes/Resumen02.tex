\documentclass[a4,11pt]{aleph-notas}
% Se puede ver la documentación aquí: 
% https://github.com/alephsub0/LaTeX_aleph-notas

% -- Paquetes adicionales
\usepackage{enumitem}
\usepackage{aleph-comandos}
\usepackage{booktabs}


% -- Datos 
\institucion{Escuela de Ciencias Físicas y Matemática}
\carrera{Medicina Veterinaria}
\asignatura{Matemática I*}
\tema{Resumen no. 2: Conceptos básicos de teoría de conjuntos}
\autor{Andrés Merino}
\fecha{Semestre 2024-2}

\logouno[0.14\textwidth]{Logos/logoPUCE_04_ac}
\definecolor{colortext}{HTML}{0030A1}
\definecolor{colordef}{HTML}{0030A1}
\fuente{montserrat}


% -- Comandos adicionales
\setlist[enumerate]{label=\roman*.}


\begin{document}

\encabezado

%%%%%%%%%%%%%%%%%%%%%%%%%%%%%%%%%%%%%%%
\section{Conceptos básicos de teoría de conjuntos}
%%%%%%%%%%%%%%%%%%%%%%%%%%%%%%%%%%%%%%%

\begin{defi}[Conjunto]
    Un \textbf{conjunto} es una colección bien definida de elementos, que se denota por letras mayúsculas como $A$, $B$, $C$. Los elementos de un conjunto se representan por letras minúsculas, por ejemplo, $x \in A$ significa que $x$ es un elemento de $A$.
\end{defi}

\begin{defi}[Igualdad de conjuntos]
    Dos conjuntos $A$ y $B$ son \textbf{iguales} si y solo si tienen exactamente los mismos elementos.
\end{defi}


\begin{defi}[Conjunto vacío]
    El \textbf{conjunto vacío}, denotado por $\emptyset$, es un conjunto que no tiene elementos.
\end{defi}

\begin{defi}[Conjunto universal]
    El \textbf{conjunto universal}, denotado por $U$, es el conjunto que contiene a todos los elementos de interés en un contexto dado.
\end{defi}

\begin{defi}[Subconjunto]
    Un conjunto $A$ es un \textbf{subconjunto} de un conjunto $B$, denotado $A \subseteq B$, si cada elemento de $A$ también es un elemento de $B$, es decir,
    \[
        x \in A \implies x \in B.
    \]
\end{defi}

\begin{teo}
    Para cualquier conjunto $A$, se cumple que $\emptyset \subseteq A$ y $A \subseteq U$.
\end{teo}

%%%%%%%%%%%%%%%%%%%%%%%%%%%%%%%%%%%%%%%
\section{Operaciones con conjuntos}
%%%%%%%%%%%%%%%%%%%%%%%%%%%%%%%%%%%%%%%

\begin{advertencia}
    Material recomendado para esta sección: \href{https://math24.net/set-operations-venn-diagrams.html}{Set Operations and Venn Diagrams}
\end{advertencia}

\begin{defi}[Diagrama de Venn]
    Un \textbf{diagrama de Venn} es una representación gráfica de conjuntos que consiste en círculos que se superponen para mostrar todas las posibles relaciones entre los conjuntos.
\end{defi}

\begin{defi}[Unión de conjuntos]
    La \textbf{unión de dos conjuntos} $A$ y $B$, denotada $A \cup B$, es el conjunto de todos los elementos que están en $A$, en $B$, o en ambos; es decir:
    \[
        A \cup B = \{x : x \in A \lor x \in B\}.
    \]
\end{defi}

\begin{teo}
    Para cualquier conjunto $A$, se cumple que $\emptyset \cup A = A$ y $U \cup A = U$.
\end{teo}

\begin{defi}[Intersección de conjuntos]
    La \textbf{intersección de dos conjuntos} $A$ y $B$, denotada $A \cap B$, es el conjunto de todos los elementos que están tanto en $A$ como en $B$; es decir:
    \[
        A \cap B = \{x : x \in A \land x \in B\}.
    \]
\end{defi}

\begin{teo}
    Para cualquier conjunto $A$, se cumple que $\emptyset \cap A = \emptyset$ y $U \cap A = A$.
\end{teo}

\begin{defi}[Diferencia de conjuntos]
    La \textbf{diferencia de conjuntos} $A$ y $B$, denotada $A \setminus B$, es el conjunto de todos los elementos que están en $A$ pero no en $B$; es decir:
    \[
        A \setminus B = \{x : x \in A \land x \notin B\}.
    \]
\end{defi}

\begin{defi}[Complemento de un conjunto]
    El \textbf{complemento de un conjunto} $A$ respecto de un conjunto universal $U$, denotado $A^c$, es el conjunto de todos los elementos que no están en $A$, es decir,
    \[
        A^c = \{x : x \in U \land x \notin A\}.
    \]
\end{defi}

\begin{teo}
    Para cualquier conjunto $A$, se cumple que $(A^c)^c = A$.
\end{teo}

\begin{defi}[Producto cartesiano]
    El \textbf{producto cartesiano} de dos conjuntos $A$ y $B$, denotado $A \times B$, es el conjunto de todos los pares ordenados $(a, b)$ donde $a \in A$ y $b \in B$; es decir,
    \[
        A \times B = \{(a, b) : a \in A \land b \in B\}.
    \]
\end{defi}


%%%%%%%%%%%%%%%%%%%%%%%%%%%%%%%%%%%%%%%%
\section{Cardinalidad}
%%%%%%%%%%%%%%%%%%%%%%%%%%%%%%%%%%%%%%%%

\begin{defi}[Cardinalidad]
    La \textbf{cardinalidad} de un conjunto $A$, denotada por $|A|$, es el número de elementos en $A$.
\end{defi}

\begin{teo}
    Dados dos conjuntos $A$ y $B$, se cumple que $|A \cup B| = |A| + |B| - |A \cap B|$.
\end{teo}

\begin{teo}
    Dados dos conjuntos $A$ y $B$, se cumple que $|A \times B| = |A| \cdot |B|$.
\end{teo}


%%%%%%%%%%%%%%%%%%%%%%%%%%%%%%%%%%%%%%%%
\section{Conjuntos numéricos}
%%%%%%%%%%%%%%%%%%%%%%%%%%%%%%%%%%%%%%%%

\begin{defi}[Conjunto de los números naturales]
    El \textbf{conjunto de los números naturales}, denotado por $\N$, es el conjunto:
    \[
        \N = \{0, 1, 2, 3, 4, \ldots\}.
    \]
\end{defi}

\begin{defi}
    El \textbf{conjunto de los números enteros}, denotado por $\Z$, es el conjunto:
    \[
        \Z = \{\ldots, -3, -2, -1, 0, 1, 2, 3, \ldots\}.
    \]
\end{defi}

\begin{defi}
    El \textbf{conjunto de los números racionales}, denotado por $\Q$, es el conjunto de todos los números que pueden ser expresados como una fracción de dos enteros, es decir:
    \[
        \Q = \left\{ \frac{a}{b} : a, b \in \Z, b \neq 0 \right\}.
    \]
\end{defi}

\begin{defi}
    El \textbf{conjunto de los números reales}, denotado por $\R$, es el conjunto de todos los números que pueden ser representados en la recta numérica.
\end{defi}

\begin{teo}
    Se tiene que
    \[
        \N \subseteq \Z \subseteq \Q \subseteq \R.
    \]
\end{teo}

\begin{defi}[Intevalos]
    Dados dos números reales $a$ y $b$, con $a < b$, se definen los siguientes intervalos:
    \begin{itemize}
        \item $\left]a, b\right[ = \{x : a < x < b\}$.
        \item $[a, b] = \{x : a \leq x \leq b\}$.
        \item $\left[a, b\right[ = \{x : a \leq x < b\}$.
        \item $\left]a, b\right] = \{x : a < x \leq b\}$.
        \item $\left]-\infty, a\right[ = \{x : x < a\}$.
        \item $\left]-\infty, a\right] = \{x : x \leq a\}$.
        \item $\left[a, +\infty\right[ = \{x : x > a\}$.
        \item $\left[a, +\infty\right] = \{x : x \geq a\}$.
    \end{itemize}
\end{defi}
    




\end{document}