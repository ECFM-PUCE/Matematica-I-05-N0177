\documentclass[a4,11pt]{aleph-notas}
% Se puede ver la documentación aquí: 
% https://github.com/alephsub0/LaTeX_aleph-notas

% -- Paquetes adicionales
\usepackage{enumitem}
\usepackage{aleph-comandos}
\usepackage{booktabs}


% -- Datos 
\institucion{Escuela de Ciencias Físicas y Matemática}
\carrera{Medicina Veterinaria}
\asignatura{Matemática I*}
\tema{Resumen no. 2: Conceptos básicos de teoría de conjuntos}
\autor{Andrés Merino}
\fecha{Semestre 2024-2}

\logouno[0.14\textwidth]{Logos/logoPUCE_04_ac}
\definecolor{colortext}{HTML}{0030A1}
\definecolor{colordef}{HTML}{0030A1}
\fuente{montserrat}


% -- Comandos adicionales
\setlist[enumerate]{label=\roman*.}


\begin{document}

\encabezado

%%%%%%%%%%%%%%%%%%%%%%%%%%%%%%%%%%%%%%%%
\section{Conceptos básicos de teoría de conjuntos}
%%%%%%%%%%%%%%%%%%%%%%%%%%%%%%%%%%%%%%

\begin{defi}[Conjunto]
    Un \textbf{conjunto} es una colección bien definida de elementos, que se denota por letras mayúsculas como $A$, $B$, $C$. Los elementos de un conjunto se representan por letras minúsculas, por ejemplo, $x \in A$ significa que $x$ es un elemento de $A$.
\end{defi}

\begin{defi}[Igualdad de conjuntos]
    Dos conjuntos $A$ y $B$ son \textbf{iguales} si y solo si tienen exactamente los mismos elementos.
\end{defi}

\begin{defi}[Subconjunto]
    Un conjunto $A$ es un \textbf{subconjunto} de un conjunto $B$, denotado $A \subseteq B$, si cada elemento de $A$ también es un elemento de $B$, es decir,
    \[
        x \in A \implies x \in B.
    \]
\end{defi}

\begin{defi}[Unión de conjuntos]
    La \textbf{unión de dos conjuntos} $A$ y $B$, denotada $A \cup B$, es el conjunto de todos los elementos que están en $A$, en $B$, o en ambos; es decir:
    \[
        A \cup B = \{x : x \in A \lor x \in B\}.
    \]
\end{defi}

\begin{defi}[Intersección de conjuntos]
    La \textbf{intersección de dos conjuntos} $A$ y $B$, denotada $A \cap B$, es el conjunto de todos los elementos que están tanto en $A$ como en $B$; es decir:
    \[
        A \cap B = \{x : x \in A \land x \in B\}.
    \]
\end{defi}

\begin{defi}[Diferencia de conjuntos]
    La \textbf{diferencia de conjuntos} $A$ y $B$, denotada $A \setminus B$, es el conjunto de todos los elementos que están en $A$ pero no en $B$; es decir:
    \[
        A \setminus B = \{x : x \in A \land x \notin B\}.
    \]
\end{defi}

\begin{defi}[Complemento de un conjunto]
    El \textbf{complemento de un conjunto} $A$ respecto de un conjunto universal $U$, denotado $A^c$, es el conjunto de todos los elementos que no están en $A$, es decir,
    \[
        A^c = \{x : x \in U \land x \notin A\}.
    \]
\end{defi}





\end{document}