\documentclass[a4,11pt]{aleph-notas}
% Se puede ver la documentación aquí: 
% https://github.com/alephsub0/LaTeX_aleph-notas

% -- Paquetes adicionales
\usepackage{enumitem}
\usepackage{aleph-comandos}
\usepackage{booktabs}


% -- Datos 
\institucion{Escuela de Ciencias Físicas y Matemática}
\carrera{Medicina Veterinaria}
\asignatura{Matemática I}
\tema{Resumen no. 5: Ecuaciones lineales y cuadráticas}
\autor{Andrés Merino}
\fecha{Semestre 2024-2}

\logouno[0.14\textwidth]{Logos/logoPUCE_04_ac}
\definecolor{colortext}{HTML}{0030A1}
\definecolor{colordef}{HTML}{0030A1}
\fuente{montserrat}


% -- Comandos adicionales
\setlist[enumerate]{label=\roman.}


\begin{document}

\encabezado

%%%%%%%%%%%%%%%%%%%%%%%%%%%%%%%%%%%%%%
\section{Ecuaciones lineales}
%%%%%%%%%%%%%%%%%%%%%%%%%%%%%%%%%%%%%%

\begin{teo}[Propiedad de la suma]
    Sean \(a, b, c \in \mathbb{R}\). Se cumple que:  
    \[
        a = b \iff a + c = b + c
    \]  
    Puedes sumar el mismo número a ambos lados de una ecuación sin alterar la igualdad.
\end{teo}

\begin{teo}[Propiedad de la resta]
    Sean \(a, b, c \in \mathbb{R}\). Se cumple que:  
    \[
        a = b \iff a - c = b - c
    \]  
    Puedes restar el mismo número a ambos lados de una ecuación sin alterar la igualdad.
\end{teo}

\begin{teo}[Propiedad del producto (sin restricciones)]
    Sean \(a, b, c \in \mathbb{R}\). Se cumple que:  
    \[
        a = b \implies a \cdot c = b \cdot c
    \]  
    Puedes multiplicar ambos lados de una ecuación por el mismo número, aunque puede que no siempre sea reversible.
\end{teo}

\begin{teo}[Propiedad del producto (con restricciones)]
    Sean \(a, b, c \in \mathbb{R}\) con \(c \neq 0\). Se cumple que:  
    \[
        a = b \iff a \cdot c = b \cdot c
    \]  
    Multiplicar por el mismo número distinto de cero en ambos lados conserva la igualdad y es reversible.
\end{teo}

\begin{teo}[Propiedad del cociente]
    Sean \(a, b, c \in \mathbb{R}\) con \(c \neq 0\). Se cumple que:  
    \[
        a = b \iff \dfrac{a}{c} = \dfrac{b}{c}
    \]  
    Dividir ambos lados de una ecuación por el mismo número distinto de cero no altera la igualdad.
\end{teo}

\begin{teo}[Propiedad del opuesto]
    Sean \(a, b \in \mathbb{R}\). Se cumple que:  
    \[
        a = b \iff -a = -b
    \]  
    El opuesto (o negativo) de ambos lados de una ecuación conserva la igualdad.
\end{teo}

\begin{teo}[Propiedad del recíproco]
    Sean \(a, b \in \mathbb{R}\) con \(a \neq 0\) y \(b \neq 0\). Se cumple que:  
    \[
    a = b \iff \dfrac{1}{a} = \dfrac{1}{b}
    \]  
    El recíproco de ambos lados de una ecuación conserva la igualdad.
\end{teo}

\begin{teo}[Propiedad de proporciones]
    Sean \(a, b, c, d \in \mathbb{R}\) con \(b \neq 0\) y \(d \neq 0\). Se cumple que:  
    \[
    \dfrac{a}{b} = \dfrac{c}{d} \iff a \cdot d = c \cdot b
    \]  
    En una proporción, el producto cruzado de los términos es igual.
\end{teo}

%%%%%%%%%%%%%%%%%%%%%%%%%%%%%%%%%%%%%%
\section{Ecuaciones cuadráticas}
%%%%%%%%%%%%%%%%%%%%%%%%%%%%%%%%%%%%%%

\begin{teo}[Propiedad del producto nulo]
    Sean \(a, b \in \mathbb{R}\). Se cumple que:  
    \[
    a \cdot b = 0 \iff (a = 0 \qlor b = 0)
    \]  
    Si el producto de dos números es cero, al menos uno de ellos debe ser igual a cero.
\end{teo}

\begin{teo}[Solución de una ecuación cuadrática simple]
    Sea \(x, y \in \mathbb{R}\) tal que \(y \geq 0\). Se cumple que:  
    \[
        x^2 = y \iff x = \pm\sqrt{y}.
    \]  
    Si el cuadrado de un número es igual a \(y\), entonces ese número es igual a la raíz cuadrada positiva o negativa de \(y\).
\end{teo}

\begin{teo}[Fórmula general para la solución de una ecuación cuadrática]
    Sea \(a, b, c, x \in \mathbb{R}\) con \(a \neq 0\). Si \(b^2 - 4ac \geq 0\), se cumple que:  
    \[
        ax^2 + bx + c = 0 \iff x = \frac{-b \pm \sqrt{b^2 - 4ac}}{2a}.
    \]  
    La solución de una ecuación cuadrática se encuentra usando la fórmula general, siempre que el discriminante (\(b^2 - 4ac\)) sea mayor o igual a cero.
\end{teo}


\end{document}