\documentclass[a4,11pt]{aleph-notas}
% Se puede ver la documentación aquí: 
% https://github.com/alephsub0/LaTeX_aleph-notas

% -- Paquetes adicionales
\usepackage{enumitem}
\usepackage{aleph-comandos}
\usepackage{booktabs}

\usepackage{amsmath}
\usepackage{pgfplots}
\pgfplotsset{compat=1.18}



% -- Datos 
\institucion{Escuela de Ciencias Físicas y Matemática}
\carrera{Medicina Veterinaria}
\asignatura{Matemática I*}
\tema{Resumen no. 6: Inecuaciones de una variable}
\autor{Alejandro Galvis}
\fecha{Semestre 2024-2}

\logouno[0.14\textwidth]{Logos/logoPUCE_04_ac}
\definecolor{colortext}{HTML}{0030A1}
\definecolor{colordef}{HTML}{0030A1}
\fuente{montserrat}


% -- Comandos adicionales
\setlist[enumerate]{label=\roman*.}
\definecolor{colcod}{RGB}{174,218,255}
\newtcolorbox{pycodigo}
    {icono=\faKeyboardO,color=colcod,postit,top=2mm,bottom=2mm}


\begin{document}

\encabezado

%%%%%%%%%%%%%%%%%%%%%%%%%%%%%%%%%%%%%%%
\section{Definición de Inecuación}
%%%%%%%%%%%%%%%%%%%%%%%%%%%%%%%%%%%%%%%

En medicina veterinaria, las \textbf{inecuaciones} son herramientas matemáticas útiles para modelar y resolver problemas relacionados con límites, rangos y condiciones óptimas en el cuidado animal. Estas desigualdades permiten determinar:

\begin{itemize}
    \item \textbf{Rangos de temperatura ideales:} Por ejemplo, garantizar que la temperatura en un ambiente para aves se mantenga entre \(20^\circ \text{C}\) y \(25^\circ \text{C}\), expresado como \(20 \leq T \leq 25\).
    
    \item \textbf{Dosis seguras de medicamentos:} Para asegurar que la concentración de un medicamento no sea tóxica, se puede expresar como \(C \leq 0.5 \, \text{mg/mL}\).
    
    \item \textbf{Criterios de peso saludable:} En animales jóvenes, el peso ideal puede establecerse como \(P \geq 5 \, \text{kg}\), asegurando un desarrollo adecuado.
    
    \item \textbf{Condiciones fisiológicas:} Las inecuaciones se utilizan para evaluar parámetros como frecuencia cardíaca o niveles de glucosa en rangos saludables, considerando valores superiores o inferiores a ciertos límites críticos.
\end{itemize}

\paragraph{Ejemplo:} Un veterinario necesita asegurar que la frecuencia cardíaca de un caballo en reposo esté entre 28 y 40 latidos por minuto, lo que puede representarse como:
\[
28 \leq f \leq 40.
\]

Estas aplicaciones prácticas permiten a los veterinarios tomar decisiones precisas y garantizar el bienestar animal.

\vspace{1pt}
\begin{defi}[Inecuación]
    Una \textbf{inecuación} es una expresión matemática que establece una relación de \textit{desigualdad} entre dos cantidades o expresiones algebraicas. En general, una inecuación se escribe como:
    \[
    f(x) \, R \, g(x),
    \]
    donde:
    \begin{itemize}
        \item \( f(x) \) y \( g(x) \) son funciones, expresiones algebraicas o constantes.
        \item \( R \) es una relación de desigualdad, que puede ser una de las siguientes:
        \[
        R \in \{\leq, <, \geq, >\}.
        \]
    \end{itemize}
    El objetivo al resolver una inecuación es determinar el conjunto de valores de la variable \( x \) que satisface la relación de desigualdad.
\end{defi}


Como se establece en su definición, una \textbf{inecuación} es una expresión matemática que establece una relación de \textit{desigualdad} entre dos cantidades o expresiones. Las más comunes son las siguientes:
\begin{itemize}
    \item \( a < b \): ``\(a\) es menor que \(b\)''.
    \item \( a \leq b \): ``\(a\) es menor o igual que \(b\)''.
    \item \( a > b \): ``\(a\) es mayor que \(b\)''.
    \item \( a \geq b \): ``\(a\) es mayor o igual que \(b\)''.
\end{itemize}

\subsection*{Inecuaciones Elementales}
\begin{itemize}
    \item \textbf{Inecuaciones Lineales}
    \begin{itemize}
        \item Forma General: $$ax + b \, R \, 0, \quad \text{con } a \neq 0.$$
        \item Ejemplo: $$2x + 3 > 0.$$
    \end{itemize}
    
    \item \textbf{Inecuaciones Cuadráticas}
    \begin{itemize}
        \item Forma General: $$ax^2 + bx + c \, R \, 0, \quad \text{con } a \neq 0.$$
        \item Ejemplo: $$x^2 - 4x + 3 \leq 0.$$
    \end{itemize}
    
    \item \textbf{Inecuaciones Racionales}
    \begin{itemize}
        \item Forma General: $$\frac{P(x)}{Q(x)} \, R \, 0, \quad \text{con } Q(x) \neq 0.$$
        \item Ejemplo: $$\frac{x - 2}{x + 3} > 0.$$
    \end{itemize}
    
    \item \textbf{Inecuaciones con Valor Absoluto}
    \begin{itemize}
        \item Forma General: $$|f(x)| \, R \, c, \quad \text{con } c \geq 0.$$
        \item Ejemplo: $$|x - 3| < 5.$$
    \end{itemize}
    
    \item \textbf{Inecuaciones Dobles}
    \begin{itemize}
        \item Forma General: $$c_1 \, R_1 \, f(x) \, R_2 \, c_2.$$
        \item Ejemplo: $$-3 < 2x + 1 \leq 5.$$
    \end{itemize}
\end{itemize}


%%%%%%%%%%%%%%%%%%%%%%%%%%%%%%%%%%%%%%%
\section{¿Cómo resolver inecuaciones elementales?}
%%%%%%%%%%%%%%%%%%%%%%%%%%%%%%%%%%%%%%%

\subsection*{1. Inecuaciones Lineales}

\subsubsection*{Fácil}
Resolver:
$$
3x + 5 \leq 11.
$$

\textbf{Resolución:}
\begin{enumerate}
    \item Restamos \(5\) a ambos lados de la inecuación para despejar \(x\):
    $$
    3x \leq 11 - 5.
    $$
    \item Simplificamos:
    $$
    3x \leq 6.
    $$
    \item Dividimos entre \(3\) (manteniendo el sentido de la desigualdad porque estamos dividiendo por un número positivo):
    $$
    x \leq 2.
    $$
\end{enumerate}

\textbf{Solución:}
$$
x \in (-\infty, 2].
$$

\subsubsection*{Reto}
Resolver:
$$
\frac{4x - 5}{3} - \frac{x + 1}{2} > 1.
$$

\textbf{Resolución:}
\begin{enumerate}
    \item Eliminamos los denominadores multiplicando toda la inecuación por el mínimo común múltiplo de \(3\) y \(2\), que es \(6\):
    $$
    6 \cdot \frac{4x - 5}{3} - 6 \cdot \frac{x + 1}{2} > 6 \cdot 1.
    $$
    Esto da como resultado:
    $$
    2(4x - 5) - 3(x + 1) > 6.
    $$
    \item Expandimos los paréntesis:
    $$
    8x - 10 - 3x - 3 > 6.
    $$
    \item Simplificamos los términos semejantes:
    $$
    5x - 13 > 6.
    $$
    \item Sumamos \(13\) a ambos lados:
    $$
    5x > 19.
    $$
    \item Dividimos entre \(5\):
    $$
    x > \frac{19}{5}.
    $$
\end{enumerate}

\textbf{Solución:}
$$
x \in \left(\frac{19}{5}, \infty\right).
$$

\subsection*{2. Inecuaciones Cuadráticas}

\subsubsection*{Fácil}
Resolver:
$$
x^2 - 4x \leq 0.
$$

\textbf{Resolución:}
\begin{enumerate}
    \item Factorizamos el binomio:
    $$
    x(x - 4) \leq 0.
    $$
    \item Identificamos los puntos críticos (\(x\) que anulan cada factor):
    $$
    x = 0 \quad \text{y} \quad x = 4.
    $$
    \item Dividimos la recta numérica en los intervalos:
    $$
    (-\infty, 0), \quad (0, 4), \quad (4, \infty).
    $$
    \item Probamos un valor en cada intervalo para determinar el signo del producto:
    \begin{itemize}
        \item En \( (-\infty, 0) \): Probamos \(x = -1\), \( (-1)(-1 - 4) = (+)\).
        \item En \( (0, 4) \): Probamos \(x = 2\), \( (2)(2 - 4) = (-)\).
        \item En \( (4, \infty) \): Probamos \(x = 5\), \( (5)(5 - 4) = (+)\).
    \end{itemize}
    \item Seleccionamos los intervalos donde el producto es menor o igual a cero:
    $$
    x \in [0, 4].
    $$
\end{enumerate}

\textbf{Solución:}
$$
x \in [0, 4].
$$

\subsubsection*{Reto}
Resolver la inecuación:
$$
x^2 - 6x + 5 > 0.
$$

\subsection*{Resolución}
\begin{enumerate}
    \item Factorizamos el trinomio:
    $$
    x^2 - 6x + 5 = (x - 1)(x - 5).
    $$
    \item Identificamos los puntos críticos:
    $$
    x = 1, \, x = 5.
    $$
    \item Dividimos la recta numérica en los intervalos determinados por los puntos críticos:
    $$
    (-\infty, 1), \quad (1, 5), \quad (5, \infty).
    $$
    \item Analizamos los signos de \( (x - 1)(x - 5) \) en cada intervalo mediante la siguiente tabla:
\end{enumerate}

\begin{center}
\begin{tabular}{|c|c|c|c|c|c|}
\hline
\textbf{Intervalo}       & \textbf{Valor de prueba} & \(\mathbf{x - 1}\) & \(\mathbf{x - 5}\) & \textbf{Producto \((x - 1)(x - 5)\)} & \textbf{Signo} \\
\hline
\( (-\infty, 1) \)       & \( x = 0 \)              & \(-\)              & \(-\)              & \( (+) \)                             & \(+\)          \\
\hline
\( (1, 5) \)             & \( x = 3 \)              & \(+\)              & \(-\)              & \( (-) \)                             & \(-\)          \\
\hline
\( (5, \infty) \)        & \( x = 6 \)              & \(+\)              & \(+\)              & \( (+) \)                             & \(+\)          \\
\hline
\end{tabular}
\end{center}

\textbf{Solución:}
$$
x \in (-\infty, 1) \cup (5, \infty).
$$

%%%%%%%%%%%%%%%%%%%%%%%%%%%%%%%%%%%%%%%
\section{¿Cómo resolver inecuaciones... un poco más dificiles?}
%%%%%%%%%%%%%%%%%%%%%%%%%%%%%%%%%%%%%%%

\subsection*{1. Inecuación Racional}

Resolver la siguiente inecuación:
$$
\frac{x^2 - 4}{x - 3} \leq 0.
$$

\subsection*{2. Inecuación con Valor Absoluto}

Resolver la siguiente inecuación:
$$
|2x - 5| \geq 7.
$$



%%%%%%%%%%%%%%%%%%%%%%%%%%%%%%%%%%%%%%%
\section{Un poco de programación en Python}
%%%%%%%%%%%%%%%%%%%%%%%%%%%%%%%%%%%%%%%

%\begin{tcolorbox}[title=¡Importante!]
%    Material recomendado para esta sección acceder al siguiente enlace:\\
%    \href{https://colab.research.google.com/drive/1dtsJCBOb-2m2BGXqca7iavwkilENi69B?usp=sharing}{Notebook de Google Colab}
%\end{tcolorbox}

Material en construcción...
\begin{pycodigo}
    \textbf{Enlace:} Material recomendado para esta sección acceder a estos enlace
    \begin{itemize}
        \item 
        \href{https://colab.research.google.com/drive/1dtsJCBOb-2m2BGXqca7iavwkilENi69B?usp=sharing}{Notebook de Google Colab}
        \item 
        \href{https://colab.research.google.com/github/ECFM-PUCE/Python-Notebooks-Educativos/blob/main/Representacion-funciones-calculadora.ipynb}{Representación de funciones (calculadora)}
        \item 
        \href{https://colab.research.google.com/github/ECFM-PUCE/Python-Notebooks-Educativos/blob/main/Representacion-funciones-codigo.ipynb}{Representación de funciones (código)}
    \end{itemize}
\end{pycodigo}



\end{document}