\documentclass[a4,11pt]{aleph-notas}

% -- Paquetes adicionales
\usepackage{enumitem}
\usepackage{aleph-comandos}
\hypersetup{urlcolor=blue}

% -- Datos  
\institucion{Escuela de Ciencias Físicas y Matemática}
\carrera{Medicina Veterinaria}
\asignatura{Matemática I}
\tema{Clase invertida no. 1: Ecuación cuadrática por Factorización}
\autor{Andrés Merino}
\fecha{Semestre 2024-2}

\logouno[0.14\textwidth]{Logos/logoPUCE_04_ac}
\definecolor{colortext}{HTML}{0030A1}
\definecolor{colordef}{HTML}{0030A1}
\fuente{montserrat}

% -- Comandos adicionales
\begin{document} 

\encabezado

\section*{Introducción}

\begin{itemize}
    \item \textbf{Tema:} Resolución de Ecuaciones Cuadráticas por Factorización
    \item \textbf{Resultado de Aprendizaje:} El estudiante será capaz de resolver ecuaciones cuadráticas aplicando la técnica de factorización.
\end{itemize}

\section{Lección en casa}

\subsection{Adquisición de concepto}

Para la adquisición del concepto, se solicitará al estudiante interactuar con ChatGPT y la visualización de video, siguiendo los siguientes pasos:

\begin{enumerate}[leftmargin=*,label=\arabic*.]
    \item Visualiza el siguiente video: \href{https://youtu.be/YGwoljiY68s}{Resolver ecuaciones con la ayuda de la propiedad cero del producto}.
    \item Interactuar con ChatGPT mediante los siguientes \textit{prompts}, leyendo detenidamente el \textit{prompt} y su respuesta:
    \begin{enumerate}[label=\textit{Prompt \arabic*.},leftmargin=2.1cm]
        \item Vas a ser mi profesor de Matemáticas y me enseñarás cómo resolver ecuaciones cuadráticas. Vamos a trabajar específicamente en resolverlas por factorización. Explica de manera detallada y clara, y asegúrate de que la parte matemática sea legible. ¿Entendido?
        \item ¿Qué es una ecuación cuadrática y qué pasos se deben seguir para resolverla por factorización? No me des un ejemplo aún.
        \item Enséñame cómo resolver una ecuación cuadrática simple, por ejemplo, $x^2 - 5x + 6 = 0$, usando la técnica de factorización.
    \end{enumerate}
    \item Visualiza el siguiente video: \href{https://www.youtube.com/watch?v=VB0yZ6fxQlU}{Resolviendo una cuadrática por factorización}.
    \item Continúa la interacción con ChatGPT mediante los siguientes \textit{prompts}, leyendo detenidamente el \textit{prompt} y su respuesta:
    \begin{enumerate}[label=\textit{Prompt \arabic*.},leftmargin=2.1cm,start=4]
        \item Dame un ejemplo con una ecuación cuadrática que pueda resolverse usando factorización.
        \item Plantéame un ejercicio de ecuación cuadrática que pueda resolverse usando factorización.
    \end{enumerate}
    \item Revisa el siguiente artículo: \href{https://es.khanacademy.org/math/algebra/x2f8bb11595b61c86:quadratic-functions-equations/x2f8bb11595b61c86:quadratics-solve-factoring/a/solving-quadratic-equations-by-factoring}{Resolver cuadráticas por factorización}.
    \item Continuar la interacción con ChatGPT para resolver dudas y profundizar sobre las explicaciones.
    \item Realiza el cuestionario del aula virtual.
\end{enumerate}

\end{document}