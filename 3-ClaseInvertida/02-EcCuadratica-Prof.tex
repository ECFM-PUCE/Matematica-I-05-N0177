\documentclass[a4,11pt]{aleph-notas}

% -- Paquetes adicionales
\usepackage{enumitem}
\usepackage{aleph-comandos}
\usepackage{aleph-moodle}
\hypersetup{urlcolor=blue}

% -- Datos   
\institucion{Escuela de Ciencias Físicas y Matemática}
\carrera{Medicina Veterinaria}
\asignatura{Matemática I}
\tema{Clase invertida no. 1: Ecuación cuadrática por Factorización}
\autor{Andrés Merino}
\fecha{Semestre 2024-2}

\logouno[0.14\textwidth]{Logos/logoPUCE_04_ac}
\definecolor{colortext}{HTML}{0030A1}
\definecolor{colordef}{HTML}{0030A1}
\fuente{montserrat}

% -- Comandos adicionales
\begin{document}

\encabezado

\section*{Introducción}

\begin{itemize}
    \item \textbf{Tema:} Resolución de Ecuaciones Cuadráticas por Factorización
    \item \textbf{Resultado de Aprendizaje:} El estudiante será capaz de resolver ecuaciones cuadráticas aplicando la técnica de factorización.
\end{itemize}

\section{Lección en casa}

\subsection{Adquisición de concepto}

Para la adquisición del concepto, se solicitará al estudiante interactuar con ChatGPT y la visualización de video, siguiendo los siguientes pasos:

\begin{enumerate}[leftmargin=*,label=\arabic*.]
    \item Visualiza el siguiente video: \href{https://youtu.be/YGwoljiY68s}{Resolver ecuaciones con la ayuda de la propiedad cero del producto}.
    \item Interactuar con ChatGPT mediante los siguientes \textit{prompts}, leyendo detenidamente el \textit{prompt} y su respuesta:
    \begin{enumerate}[label=\textit{Prompt \arabic*.},leftmargin=2.1cm]
        \item Vas a ser mi profesor de Matemáticas y me enseñarás cómo resolver ecuaciones cuadráticas. Vamos a trabajar específicamente en resolverlas por factorización. Explica de manera detallada y clara, y asegúrate de que la parte matemática sea legible. ¿Entendido?
        \item ¿Qué es una ecuación cuadrática y qué pasos se deben seguir para resolverla por factorización? No me des un ejemplo aún.
        \item Enséñame cómo resolver una ecuación cuadrática simple, por ejemplo, $x^2 - 5x + 6 = 0$, usando la técnica de factorización.
    \end{enumerate}
    \item Visualiza el siguiente video: \href{https://www.youtube.com/watch?v=VB0yZ6fxQlU}{Resolviendo una cuadrática por factorización}.
    \item Continúa la interacción con ChatGPT mediante los siguientes \textit{prompts}, leyendo detenidamente el \textit{prompt} y su respuesta:
    \begin{enumerate}[label=\textit{Prompt \arabic*.},leftmargin=2.1cm,start=4]
        \item Dame un ejemplo con una ecuación cuadrática que pueda resolverse usando factorización.
        \item Plantéame un ejercicio de ecuación cuadrática que pueda resolverse usando factorización.
    \end{enumerate}
    \item Revisa el siguiente artículo: \href{https://es.khanacademy.org/math/algebra/x2f8bb11595b61c86:quadratic-functions-equations/x2f8bb11595b61c86:quadratics-solve-factoring/a/solving-quadratic-equations-by-factoring}{Resolver cuadráticas por factorización}.
    \item Continuar la interacción con ChatGPT para resolver dudas y profundizar sobre las explicaciones.
    \item Realiza el cuestionario del aula virtual.
\end{enumerate}

\subsection{Personalización de la actividad}

Se la consigue solicitando al estudiante continuar la interacción hasta que sienta que ha asimilado el concepto.

\subsection{Solventación de dudas}

En caso de tener dudas sobre el tema, se solicitará al estudiante interactuar con GhatGPT.

\subsection{Micro-tarea}

Para realizar un seguimiento de la actividad, se solicitará al estudiante copiar el enlace del chat como evidencia del proceso. Adicionalmente, se le pedirá realizar el cuestionario del aula virtual. El cuestionario se encuentra detallado en el Anexo.

\section{Tareas en clase}

\subsection{Visión conjunta}

Relacionar las actividades realizadas en casa con la práctica en clase. Se revisarán las soluciones a las ecuaciones cuadráticas resueltas por factorización y se discutirán diferentes tipos de factores.

\subsection{Retroalimentación}

Proveer retroalimentación sobre los resultados de las micro-tareas y cuestionarios realizados en casa.

\subsection{Actividad de aplicación}

Resolver los siguientes ejercicios:

\begin{enumerate}
    \item Resuelve la ecuación cuadrática: $2x^2 - 8x = 0$.
    \item Resuelve la ecuación cuadrática: $x^2 - 9x + 14 = 0$.
    \item Resuelve la ecuación cuadrática: $x^2 + 6x - 16 = 0$.
\end{enumerate}

\subsection{Micro-evaluación}

No cuenta con microevaluación

\section{Anexo}

\begin{quiz}{Ecuación cuadrática}

%%%%%%%%%%%%%%%%%%%%%%%%%%%%%%%%%%%%%%%
\begin{numerical}[]%
    % - Indentificador
    {Ecuación cuadrática por Factorización - 1}
    % - Enunciado
    Encuentra los valores de $x$. Ingresa la solución menor:
    \[
        2 x^{2} + 8 x - 24 = 0
    \]
    \item[] -6
\end{numerical}

\end{quiz}

Se plantea un cuestionario de 50 preguntas de este tipo.

\begin{quiz}{Clase Invertida Ecuación cuadrática por Factorización}
    
\begin{essay}[response format=text, response field lines=5]%
    % - Identificador
    {ClaseInvertida-Chat}
    % - Enunciado
    Copia el enlace del chat con ChatGPT como evidencia de la actividad realizada en casa.
    \item Acceder al enlace.
\end{essay}

\begin{essay}[response format=text, response field lines=5]%
    % - Identificador
    {ClaseInvertida-Sol}
    % - Enunciado
    ¿Alguna pregunta que ChatGPT no te supo responder?
    \item Solo para registro.
\end{essay}

\begin{essay}[response format=text, response field lines=5]%
    % - Identificador
    {ClaseInvertida-Dudas}
    % - Enunciado
    ¿Qué dudas tienes sobre la resolución de ecuaciones cuadráticas por factorización?
    \item Solo para registro.
\end{essay}



\end{quiz}


\end{document}